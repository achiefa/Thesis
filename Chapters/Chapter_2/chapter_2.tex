\chapter{Polarised Inclusive and Semi-inlcusive Deeply-Inelastic Scattering}
\label{ch:2}

Since the discovery of the so-called "spin crisis" in the 80s \cite{Leader_Anselmino}, several efforts have been applied to the measurements of polarised inclusive deeply inelastic processes in lepton-nucleon collisions at labs like CERN, SLAC, DESY and Jefferson Lab. Deep inelastic lepton-hadron scattering (DIS) has set the stage for our present understanding of the sub-structure of elementary particles. The idea of point-like constituents in hadrons consolidated after the discovery of the Bjorken scaling, leading to the invention of the Parton Model (PM). Throughout the subsequent long period, DIS continued to shed light on the hadron structure, strengthening the idea of the existence of missing constituents then identified as gluons, bringing together all the different ingredients of the hadron picture into a coherent dynamical theory of quarks and gluon - Quantum Chromodynamics (QCD). A complementary and equally important insight into the structure of the nucleon was then provided by polarised DIS, which allows for the investigation of the spin structure of hadrons. In fact, considerable effort has been spent after the results of the European Muon Collaboration in the 1980. Although the interpretation of the results were not completely correct, the measurement clearly indicated that only a part of the proton's spin is carried by the valence quark, leading to the so-called "spin crisis" in constituent quark model\footnote{In particular, the experiment staeted the failure of the sum rules due to Ellis and Jaffe based on the assumption that the contribution from strange quarks to $g_1$ is negligible.}. This crisis was then resolved by the inclusion of other source in the decomposition of the proton spin, such as the sea quarks and the gluon spin contributions, the latter heuristically predicted in the PM, then naturally introduced by QCD.\par
Contrary to the constituent quark model, whose description of the hadron relies on a static picture of its structure, the PM introduced much more dynamics in the description, providing the basis for a more general field-theoretic framework. The improvements brought by QCD, together with it high prediction power, in the description of high energy processes at colliders strengthened the reliability of the theory, eventually adopted to describe strong interactions.\par
This chapter is devoted to the introduction of the theoretical foundations necessary to describe the physical processes discussed and examinated in this Thesis. We first derive the results in the PM framework, since it provides an intuitive approach the quantities that enter the analysis. Then, we will see how the QCD affects the PM predictions, focusing on the two important modifications that QCD introduces - the gluon contribution and the DGLAP evolution equations. The results shown in this chapter will be used in Chapter \ref{ch:3}.

%___________________________________________________________________________
\section{Theoretical framework}
The inclusive inelastic scattering of a polarised lepton beam off a polarised nucleon target can be sketched as
%%
\begin{equation}
    \ell(k,s) + N(P, S) \longrightarrow \ell'(k',s') + X(P_X) \,,
    \label{eq:DIS}
\end{equation}
%%
where $P$ is the four-momentum of the nucleon target and $k,k'$ are the four-momenta of the incoming ($\ell$) and outgoing $(\ell')$ leptons, respectively. The additional labels $S,s$ and $s'$ indicate the spin along the quantization direction of the target, the incoming and the outgoing leptons. Finally, $X$ indicates the sum of all the undetected final states with aggregative momentum $P_X$. Only in the limit of large momentum transfer this reaction can be referred to as Deep Inelastic Scattering (DIS), as we shall see. In this thesis, we will assume that the process is dominated by processes with only one photon exchanged, since the ranges of values in $Q^2$ provided by the present polarised data never probe the region in which weak corrections become important\footnote{It is possible to imagine a process with multiple photon exchanged (for example, see \cite{leader_predazzi_1996}). However, these higher perturbative orders in QED are negligible with respect to the amplitudes of QCD reactions that take place inside the hadron. Hence, we shall neglect such QED corrections henceforth.}. The Feynman diagram of the of this process is show in Fig.\ref{fig:DIS_Feynamn}.
%%
\begin{figure}[h]
  \centering
  \includegraphics[width=0.5\textwidth]{DIS.pdf} 
  \caption{Feynman diagram for deep inelastic lepton-hadron scattering.}
  \label{fig:DIS_Feynamn}
\end{figure}
%%
The kinematics is worked out in the reference frame of the laboratory, which corresponds to the fixed-target frame in polarised experiments. The four-momenta are defined as follows
%%
\begin{equation}
    \begin{split}
        &k^{\mu} \;=\; \left(E, \vb{k} \right), \\
        &k'^{\mu} \;=\; \left(E', \vb{k'} \right), \\
        &P^{\mu} \;=\; \left(M, \vb{0} \right),
    \end{split}
\end{equation}
%%
where $M$ is the nucleon mass. We will make use of the following invariant variables
%%
\begin{align}
        & \nu = \frac{q \cdot P}{M} = E - E' \;\; \textrm{is the energy lost by incoming lepton in the nucleon's rest frame.}
        \label{eq:ch2:nu}\\
        & Q^2 = - q^2 \;\; \textrm{is the virtuality of the exchanged photon.}
        \label{eq:ch2:Q2}\\
        & x=\frac{Q^2}{2M\nu} \;\; \textrm{is the Bjorken variable.}
        \label{eq:ch2:x}\\
        & y = \frac{q\cdot P}{k \cdot P} = \frac{\nu}{E} \; \; \textrm{is the energy fraction lost by the incoming lepton}
        \label{eq:ch2:y}\\
        & W^2 = \qty(P + q)^2 = M^2 + Q^2 \frac{1-x}{x} \;\; \textrm{is the invariant mass of the system X.}
        \label{eq:ch2:W2}
\end{align}
%%
The process  is said to be "deep" when $Q^2 \gg M^2$ (\textit{i.e.} large momentum transfer), whereas the attribute "inelastic" is appropriate when $W^2 \gg M^2$ (\textit{i.e.} the hadron target is much inelastically excited). The amplitude for a single process representing Fig.\ref{fig:DIS_Feynamn}, with a generic hadron state $X$, is given by
%%
\begin{equation}
    \mathcal{M} = (-i e) \, \overline{u}(k',s') \, \gamma^{\mu}\, u(k,s) \, \frac{-i}{q^2} \, (-i e)\bra{X} j_{\mu}^{em} \qty(0) \ket{P,S} \,.
    \label{eq:ch2:amplitude}
\end{equation}
%%
Being a total-inclusive process, no particular particles in the final state are detected. Hence, the total cross-section must take into account all the possible out-states, that is a sum over all the possible states $X$. For this reason, only two kinematic variables, among those listed in Eqs.~[ref{eq:ch2:nu}-ref{eq:ch2:W2}] are required for providing a complete description of the process. The differential cross-section for the process $2 \rightarrow 1 + n_x$ in the FT frame is given in Appendix and in our case takes the form
%%
\begin{equation}
    d\sigma = \frac{1}{4 \qty|\vb{k}| M} \sum_{X} \qty| \mathcal{M} \qty( \ell N \rightarrow \ell'X ) |^2 \, \widetilde{d k'} \,  d\Phi_X \, ,
    \label{eq:ch2:dsigma}
\end{equation}
%%
where
%%
\begin{equation}
        d \Phi_x = \qty(2\pi)^4 \delta^{(4)} \qty( k + P - k' - P_X ) \tilde{dP_X}  
\end{equation}
\\
and $\widetilde{dP_X}$ is the Lorentz invariant phase-space measure defined in appendix \eqref{eq:cs:inv_phase}
\\
\begin{equation}
    \widetilde{d\, P_X} = \frac{d^3 \, P_X}{2 E_X (2\pi)^3} \,.
\end{equation}
%%
An identical definition holds also for $\widetilde{d k'}$.  Since the measures in inclusive DIS concern only the final lepton $\ell'$, we can integrate over the entire momentum space in the hadron state and thus rewrite eq. \eqref{eq:ch2:dsigma} as
%%
\begin{equation}
    \begin{split}
        d\sigma &= \frac{1}{4 \qty|\vb{k}| M} \sum_{X} \qty| \mathcal{M} \qty( \ell N \rightarrow \ell'X ) |^2 \tilde{d k'} \int d\Phi_X\\
        & = \frac{1}{4 \qty|\vb{k}| M} \sum_{X} \frac{e^4}{q^4} \qty[ \overline{u}_{s'}(k') \, \gamma^{\mu}\, u_s(k)] \qty[\overline{u}_{s}(k) \,\gamma^{\nu}\, u_{s'}(k') ] \; \frac{d^3 \, k'}{2 E' (2\pi)^3} \times \\
        & \hspace{10mm} \int d\Phi_X  \bra{X} j_{\mu}^{em} \qty(0) \ket{P,S} \; \bra{P,S} j_{\nu}^{em} \qty(0) \ket{X}  \\
        & = \frac{\alpha_{\textrm{em}}^2}{2 M \qty|\vb{k}| q^4} L_{\mu\nu} W^{\mu\nu} \qty|\vb{k}'| dE' d\Omega \,,
        \label{eq:ch2:dsigma2}
    \end{split}
\end{equation}
%%
where in the last equality I used $d^3 \, k' = \qty| \vb{k}' |^2 d|\vb{k}'| d\Omega$. The tensors $L_{\mu\nu}$ and $W_{\mu\nu}$ are defined as follows
\\
\begin{align}
    &L_{\mu\nu} = \qty[ \overline{u}_{s'}(k') \, \gamma^{\mu}\, u_s(k)] \qty[\overline{u}_{s}(k) \,\gamma^{\nu}\, u_{s'}(k') ]
    \label{eq:ch2:lep_tens} \\
    &W_{\mu\nu} = \frac{1}{2\pi} \sumint_{X}  d\Phi_X  \bra{X} j_{\mu}^{em} (0) \ket{P,S}  \bra{P,S} j_{\nu}^{\dag em} (0) \ket{X}  
    \label{eq:ch2:had_tens}
\end{align}
\\
In the DIS limit, we can make use of the following approximations
\\
\begin{align}
    &k^2 = k'^2 \approx 0 \hspace{2mm} \Rightarrow \hspace{2mm} E \approx \qty| \vb{k} |, \hspace{2mm} E' \approx \qty| \vb{k'}| 
    \label{eq:ch2:lep_approx}\\
    &Q^2 = - \qty(k - k')^2 \approx  2EE' \qty(1 - \cos \theta) = 4EE' \sin^2 \qty(\frac{\theta}{2}) \,,
    \label{eq:ch2:Q2_approx}
\end{align}
\\
where $\theta$ is the angle between the incoming and scattered leptons (\red{Forse da inserire anche prima siccome compare $\Omega$ in $d\sigma$.}). Inserting eqs. \eqref{eq:ch2:lep_approx}, \eqref{eq:ch2:Q2_approx} in eq. \eqref{eq:ch2:dsigma2} one obtains 
\\
\begin{equation}
    \frac{d^2\,\sigma}{dE' \, d\Omega} = \frac{\alpha_{\textrm{em}}^2}{2MQ^4} \frac{E'}{E} L_{\mu\nu}W^{\mu\nu} = \frac{\alpha_{\textrm{em}}^2 E'}{s\, Q^4} L_{\mu\nu}W^{\mu\nu}\,,
    \label{eq:ch2:dsigma3}
\end{equation}
\\
where $s = (k + P)^2 \approx 2 M E$ is the Mandlestam variable. This expression gives the differential cross-section as a function of the energy of the outgoing lepton $E'$ and the scattered solid angle $\Omega$, both evaluated in the FT frame. It is possible to express eq. \eqref{eq:ch2:dsigma3} in terms of the Lorentz invariant dimensionless variables eqs. \eqref{eq:ch2:x},\eqref{eq:ch2:y} by performing a transformation $(E', \theta) \rightarrow (x,y)$, which leads to
\\
\begin{equation}
    \frac{d^3 \, \sigma}{dx \, dy \, d\varphi} = \frac{\alpha_{\textrm{em}}^2 y}{2 Q^4} L_{\mu\nu} W^{\mu\nu} \,.
    \label{eq:ch2:dsigma4}
\end{equation}
\\
It is worth noting that, in computing eq. \eqref{eq:ch2:dsigma4}, the sum over the lepton and hadron spin has not been taken into account. In principle, the lepton and hadron tensors can contain both symmetric and antisymmetric contributions, but we shall see that, in the polarised case, only the antisymmetric component survives.

%___________________________________________________________________________
\subsection*{Leptonic tensor}
The matrix element described by the lepton tensor can be decomposed into a symmetric and an antisymmetric part under $\mu \leftrightarrow \nu$ interchange. First, from Appendix \eqref{eq:spin:projectors} we see that a polarised spinor $u \qty(k, \, s)$ with polarisation vector $s$ satisfies the identity 
\\
\begin{equation}
    u \qty(k, \, s) \overline{u} \qty(k,\,s) = \frac{1}{2} (1 - \gamma_5 \slashed{s})  \qty(\slashed{p} + m)\,.
\end{equation}
\\
After summing over the spin of the final lepton $s'$, the lepton tensor can be written as a trace over gamma matrices
\\
\begin{equation}
    \begin{split}
        L_{\mu \nu} \qty(k,\,s;\, k') &= \frac{1}{2} \Trace \qty[ \qty(\slashed{k}' + m) \gamma_{\mu} \qty(1 - \gamma_5 \slashed{s}) \qty(\slashed{k} + m) \gamma_{\nu} ]\\
        & = \frac{1}{2} \qty[ \Trace \qty[ \slashed{k} \gamma_{\mu} \slashed{k'} \gamma_{\nu} ] + m_{\ell} s^{\lambda} \Trace \qty[ \gamma_{\lambda} \gamma_{\nu} \gamma_{\rho} \gamma_{\mu} \gamma_{5} ] \qty( k - k' )^{\rho} ]\\
        & = L_{\mu \nu}^{(S)} + i L_{\mu \nu}^{(A)} \,,
    \end{split}
\end{equation}
\\
where $L_{\mu \nu}^{(S)}$ and $L_{\mu \nu}^{(A)}$ are the symmetric and antisymmetric parts defined as 
\\
\begin{equation}
    L_{\mu \nu}^{(S)} \qty(k;\, k') = \frac{1}{2} \Trace \qty[ \slashed{k} \gamma_{\mu} \slashed{k'} \gamma_{\nu} ] = 2 \Bigl[k'_{\mu} k_{\nu} + k_{\mu} k_{\nu}' - g_{\mu \nu} \qty( k \cdot k' ) \Bigr]
\end{equation}
\\
\begin{equation}
    \begin{split}
        L_{\mu \nu}^{(A)} \qty(k,\,s;\, k') & = \frac{1}{2i} m_{\ell} s^{\lambda} \Trace \qty[  \gamma_{\mu} \gamma_{\nu} \gamma_{\lambda} \gamma_{\rho} \gamma_{5} ] \qty( k - k' )^{\rho} \\
    &  = 2m_{\ell} \epsilon_{\mu \nu \lambda  4\rho } s^{\lambda} \qty( k - k' )^{\rho}\,.
    \end{split}
\end{equation}
\\ 
If the momentum of the incoming lepton is aligned to the quantization axis, from Appendix eq. \eqref{eq:spin:spin_vector} we can express the spin vector as
\\
\begin{equation}
    s^{\mu} = \frac{\lambda_{\ell}}{m_{\ell}} k^{\mu}\, .
    \label{eq:ch2:spin_vector}
\end{equation}
\\
where $\lambda_{\ell}$ is the helicity of the particle and expresses whether the particle'spin is parallel ($\lambda_{\ell} = +1/2$) or antiparallel ($\lambda_{\ell} = -1/2$) to the direction of motion. Thus, the antisymmetric component of the lepton tensor is
\\
\begin{equation}
    L_{\mu \nu}^{(A)} \qty(k,\,s;\, k') = 2\lambda_{\ell} \epsilon_{\lambda \nu \rho \mu}  q^{\rho} k^{\lambda}\,,
\end{equation}
\\
where the mass term $m_{\ell}$ has been cancelled because of eq. \eqref{eq:ch2:spin_vector}.



%________________________________
\subsection*{Hadronic tensor}
The hadronic tensor $W^{\mu \nu}$ encloses information about the proton structure and the strong interaction dynamics in the long distance limit and its analysis cannot be addressed by means of perturbative techniques. However, exploiting symmetry constraints, the tensor can be decomposed into four scalar structure functions, the unpolarised functions $F_{1,2}$ and the helicity-dependent functions $g_{1,2}$, as long as parity violation is not of concern. Usually, these functions are measured by experiments and compared with the predictions that strongly depend on the theoretical model adopted to compute the observable. As we shall see, the predictions obtained from the parton model acquired substantial corrections when QCD is introduced, as we shall see. The hadronic tensor may be decomposed into a symmetric and an antisymmetric part
%%
\\
\begin{equation}
    \begin{split}
        W_{\mu\nu} &= \frac{1}{2\pi} \sumint_{\;\;X} \tilde{d P_X} \int d^4z \, e^{iz \cdot \qty(q + P - P_{X})}\bra{X} j_{\nu}(0) \ket{P,S} \bra{P,S} j^{\dag}_{\mu}(0) \ket{X}\\
        & = \frac{1}{2\pi} \sumint_{\;\;X} \tilde{d P_X} \int d^4z \, e^{iz \cdot q}\bra{X} j_{\nu}(0) \ket{P,S} \bra{P,S} j^{\dag}_{\mu}(z) \ket{X} \\
        & = \frac{1}{2\pi} \int d^4z \, e^{i z \cdot q} \bra{P,S} j^{\dag}_{\mu}(z) j_{\mu}(0) \ket{P,S} \,.
    \end{split}
    \label{eq:ch2:hadronic_tensor}
\end{equation}
\\
In the first line of eq. \eqref{eq:chi2:hadronic_tensor} the momentum-conservation delta function is expressed in its integral representation, introducing the integral dummy-variable z. The second line is the result of the transformation of the current-operator $j^{\dag}_{\mu}(0)$ by a space-time transformation 
\\
\begin{equation}
    \bra{P,S} j^{\dag}_{\mu}(0) \ket{X} =  \bra{P,S} e^{-i\hat{P}z} \, j^{\dag}_{\mu}(0) \, e^{i\hat{P}z} \ket{X} =  e^{-i z \cdot \qty(P - P_X )}\bra{P,S} j^{\dag}_{\mu}(z) \ket{X}\,.
\end{equation}
\\
Finally, the last line followed by exploiting the closure relation $\sumint_{\;\; X} \ket*{X} \bra*{X} = 1 $. In similar fashion to the lepton case, the hadronic tensor can be decomposed into a symmetric and an antisymmetric part
\\
\begin{equation}
    W_{\mu \nu} =  W_{\mu \nu} (q, \;P ) + i W_{\mu \nu} (q, \;P ; \; S ) \,.
\end{equation}
\\
Then, we can observe:
\begin{enumerate}
    \item The hadronic tensor can be at most function of $q, P$ and the spin vector of nucleon $S$, that is $W_{\mu \nu} = W_{\mu \nu}\qty(q,P;S)$.  
    \item The electromagnetic current must be conserved, that is $\partial \cdot j = 0$, which for eq. \eqref{eq:ch2:hadronic_tensor} is replaced by $q_{\mu} W^{\mu \nu} = 0$.
    \item The hadronic tensor is hermitian, \textit{i.e.} $W_{\mu \nu} = \qty(W_{\mu \nu})^{*}$, as can be easily seen from eq. \eqref{eq:ch2:had_tens}. 
    \item We expect the antisymmetric part of hadroninc tensor to change under reverse of the nucleon's polarisation, hence the antisymmetric part mu be linear in the polarisation vector.
    \item The strong interactions are parity invariant.\footnote{
    \red{Ci sono ancora un po' di cose da capire:
        \begin{enumerate}
            \item Cosa vuole dire che le interazioni forti sono parity invariant?
            \item In che modo otteniamo la parte antisimmetrica? Leader e Predazzi, a pagina 344, citano Bjorken 1966, 1971 per motivare la forma della parte antisimmetrica, ma non ho trovato nulla du interessante. 
        \end{enumerate}
        }}
\end{enumerate}
Exploiting these constraints leads to the most general form of the hadronic tensor 
%%
\begin{equation}
  \begin{split}
  W_{\mu \nu}^{(S)} & = 2\qty(-g_{\mu \nu} + \frac{q_{\mu} q_{\nu}}{q^2}) F_1 \qty(x, Q^2) \\
  & + \frac{2 F_2 \qty(x , Q^2) }{P \cdot q} \qty(P_{\mu} - q_{\mu} \frac{P \cdot q}{q^2}) \qty(P_{\nu} - q_{\nu} \frac{P \cdot q}{q^2})\,,
  \label{eq:ch2:had_tens_final_symm}
  \end{split}
\end{equation}
%%
\begin{equation}
  \begin{split}
    W_{\mu \nu}^{(A)}  =  \epsilon_{\mu \nu \alpha \beta} \frac{q^{\alpha}2 M}{P \cdot q} \Biggl\{S^{\beta} g_1 \qty(x, Q^2) + \qty(S^{\beta} - P^{\beta} \frac{S \cdot q}{P \cdot q}) g_2 \qty(x,Q^2) \Biggr\} \,,
    \label{eq:ch2:had_tens_final_anti}
  \end{split}
\end{equation}
%%
The spin-independent coefficients $F_1,\, F_2$ and the spin-dependent ones $g_1,\,g_2$ are usually called \textit{structure functions} or \textit{form factors}. In the literature (see \textit{e.g.} Ref. \cite{leader_2001} or [\href{https://arxiv.org/abs/hep-ph/9501369}{arXiv:hep-ph/9501369}]) parametrize the scalar coefficients with $W_1 ,\, W_2$ and $G_1,\, G_2$, related with those in Eqs.[\ref{eq:ch2:had_tens_final_symm}-\ref{eq:ch2:had_tens_final_anti}] as follows:
%%
\begin{equation}
  F_1 (x, Q^2) = M W_1 (\nu, Q^2)\,, \hspace{5mm} F_2(x,Q^2) = \nu W_2 (\nu,Q^2)
\end{equation}
%%
\begin{equation}
  g_1 (x, Q^2) = M^2 \nu G_1 (\nu, Q^2) \,, \hspace{5mm} g_2 (x, Q^2) = M \nu^2 G_2 (\nu, Q^2)\,,
\end{equation}
%%
and the symmetric and antisymmetric parts of the hadronic tensor can be recast as
%%
\begin{equation}
  \begin{split}
    \frac{1}{2M} W_{\mu \nu}^{(S)} (q; \; P) & = \qty(-g_{\mu \nu} + \frac{q_{\mu} q_{\nu}}{q^2}) W_1( P \cdot q. \; q^2) \\
    & + \frac{W_2 (P\cdot q,\; q^2)}{M^2} \Biggl[ \qty(P_{\mu} - q_{\mu} \frac{P \cdot q}{q^2}) \qty(P_{\nu} - q_{\nu} \frac{P \cdot q}{q^2}) \biggr]
  \end{split}
  \label{eq:ch2:had_tens_symm}
\end{equation}
%%
\begin{equation}
  \begin{split}
    \frac{1}{2M} W_{\mu \nu}^{(A)} (q; \; P, \; S) & = \epsilon_{\mu \nu \alpha \beta} q^{\alpha} \Biggl\{ M S^{\beta} G_1 \qty(P\cdot q, Q^2) \Biggr.\\
    & +  \Biggl. \frac{G_2 \qty(P \cdot q,\;Q^2)}{M} \qty[ (P \cdot q) S^{\beta} - (S \cdot q) P^{\beta} ]\Biggr\} \,,
  \end{split}  
\end{equation}
%%
The \href{https://pdg.lbl.gov/2019/reviews/rpp2019-rev-structure-functions.pdf}{PDG} reports the expressions of the hadronic tensor, Eqs. [\ref{eq:ch2:had_tens_final_symm}-\ref{eq:ch2:had_tens_final_anti}] but with the normalisation $S^2 = - M^2$ and $S \cdot P = 0$. Is it worth remarking that the above expressions describe the pure electromagnetic process mediated by a single virtual photon. The inclusion of both neutral- and charged-current at energies near (or higher) the weak boson masses introduces parity violating terms in the decomposition of the hadronic tensor. Hence, four additional scalar functions appear, usually called $F_3, \; g_3,\; g_4$ and $g_5$. The first one multiplies a term that is spin independent and also antisymmetric, whereas the other are multiplied by terms that are spin dependent and also symmetric. Because of that the correspondence for which the symmetric part, Eq. \eqref{eq:ch2:had_tens_final_symm}, is spin independent and the antisymmetric one, Eq. \eqref{eq:ch2:had_tens_final_anti}, is spin dependent no longer hold. In fact, the spin-dependet and -independent part becomes a superposition of symmetric and antisymmetric contributions. The general decomposition for the hadronic tensor in DIS can be found in [\href{https://arxiv.org/abs/hep-ph/9401264}{arXiv:hep-ph/9401264}].\par
In the experimental data used in our analysis the momentum transfer values of the electron or muon beams do not exceed  $Q^2 \sim 100 \; GeV^2$. Hence, the contributions from a weak boson exchange can be safely neglected. However, future data may surpass the weak boson mass threshold and the most general decomposition for the hadronic tensor will become necessary to handle high energies DIS experiment, as it may happen at the Electron Ion Collider (EIC) (see \textit{e.g.} Ref. [\href{https://arxiv.org/abs/2007.08300v1}{arXiv:2007.08300v1}]).\par

\subsection{Polarised cross-section asymmetries}
The insertion of the decompositions for both the leptonic and hadronic tensor into the differential cross-section Eq \eqref{eq:ch2:dsigma4} yields to 
%%
\begin{equation}
  \frac{d^3 \, \sigma^{s,S}}{dx \, dy \, d\varphi} = \frac{\alpha_{\textrm{em}}^2 y}{2 Q^4} \Bigl[ L^{(S)}_{\mu\nu} W^{\mu\nu(S)} - L^{(A)}_{\mu\nu} W^{\mu\nu(A)}  \Bigr]\,.
  \label{eq:cross_section}
\end{equation}
%%
In order to single out the spin dependent contribution (\textit{i.e.} the antisymmetric part) one usally takes the difference of cross-sections Eq. \eqref{eq:cross_section} with opposite target spins 
%%
\begin{equation}
  \frac{d^3 \, \sigma^{s,+S}}{dx \, dy \, d\varphi} - \frac{d^3 \, \sigma^{s,-S}}{dx \, dy \, d\varphi} = - \frac{\alpha_{\textrm{em}}^2 y}{Q^4} \; L^{(A)}_{\mu\nu} W^{\mu\nu(A)} \;,
  \label{eq:diff_cs}
\end{equation}
%%
where the contributions with the opposite target spins sum together since the antisymmetric parts are linear in the target spin.\par
In our framework, we only consider longitudinally polarised leptons with spin along $(\uparrow)$ or opposite $(\downarrow)$ the direction of motion, whereas the nucleon is at rest with polarisation along $(\Uparrow)$ or opposite $\Downarrow$ an \textit{arbitrary} direction $\vb*{S}$. Hence, in the target rest frame we can parametrize the nucleon spin four-vector as
%%
\begin{equation}
  S^{\mu} = (0, \; \sin \alpha \cos\beta, \; \sin \alpha \sin \beta, \; \cos \alpha) \;,
  \label{eq:par_S}
\end{equation}
%%
and, assuming the incoming lepton to be aligned to the z-axis, we can write
%%
\begin{equation}
  \begin{split}
    & k^{\mu} = E (1,\; 0,\; 0,\; 1)\;, \\
    & k'^{\mu} = E'( 1,\; \sin\theta \cos \varphi,\; \sin\theta \sin \varphi, \; \cos \theta )\,.
    \label{eq:par_k}
  \end{split}
\end{equation} 
%%
After some algebra, the product of the two antisymmetric tensor that appears in Eq. \eqref{eq:diff_cs} looks like 
%%
\begin{equation}
  \begin{split}
    L^{\mu \nu (A)} W_{\mu \nu}^{(A)} & = - \frac{8}{\nu} \lambda_{l} \Biggl\{ g_1(x,Q^2) \Bigl[ (q \cdot k)(q \cdot S) - q^2 (k \cdot S) \Bigr] \Biggr. \\
    & +  \Biggl. g_2 (x, Q^2) q^2\Bigl[ (k \cdot S) - \frac{(q \cdot S) (k \cdot P)}{P \cdot q} \Bigr] \Biggr\}\,.
  \end{split}
\end{equation}
%%
The Lorentz products can be expressed in terms of the parametrisations Eqs. [\ref{eq:par_S}-\ref{eq:par_k}] and the cross-section difference is
%%
\begin{equation}
  \begin{split}
    \frac{d \Delta \sigma^{s,S}}{dx dy d\varphi} & = \frac{4 \alpha_{\T{em}}^2 y}{Q^2} \Biggl\{ \cos \alpha \left[ g_1(x,Q^2) \left( \frac{E}{\nu} + \frac{E'}{\nu} \cos \theta \right) - g_2(x,Q^2) \frac{2 E E'}{\nu^2} (1 - \cos \theta) \right] \Biggr. \\
    & + \Biggl. \sin\alpha \sin\theta \cos\phi \left[ g_1(x,Q^2) \frac{E'}{\nu} + g_2(x,Q^2) \frac{2EE'}{\nu^2} \right] \Biggr\}\,,
  \end{split}
  \label{eq:delta_sigma_1}
\end{equation}
%%
where $\phi = \varphi - \beta$.
The r.h.s. of Eq. \eqref{eq:delta_sigma_1} is not written in terms of the invariant variables $x$ and $y$, as the differential cross-section in the r.h.s. would require. By making use of Eqs.~[\ref{eq:ch2:x}-\ref{eq:ch2:y}] and working out the a little algebra, we can obtain the final expression for the differential cross-section difference Eq.~\eqref{eq:diff_cs}
%%
\begin{equation}
  \begin{split}
    \frac{d \Delta \sigma^{s,S}}{dx dy d\varphi} & = \frac{4 \alpha_{\T{em}}^2}{Q^2} \Biggl\{ \cos \alpha \left[ g_1(x,Q^2) \left( 2 - y - \frac{y^2 \gamma^2}{2} \right) - g_2(x,Q^2) y \gamma^2 \right] \Biggr.\\
    & + \sin \alpha \cos \phi \sqrt{1 - y - \frac{y^2 \gamma^2}{4}} \gamma \left( y g_1(x,Q^2) + 2 g_2(x,Q^2) \right) \Biggl. \Biggr\}\,,
  \end{split}
  \label{eq:cs_diff_fin_S}
\end{equation}
%%
where I have introduced the $\mathcal{O}(1/Q)$ quantity 
%%
\begin{equation}
  \gamma = \frac{2Mx}{Q} \,.
  \label{eq:gamma}
\end{equation}
The above expression relies on the assumption of longitudinally polarised leptons. Even though it would be possible to deal with transversely polarised leptons, it turns out such configuration to be phenomenologically impractical. In fact, with transversely polarised leptons Eq.~\eqref{eq:ch2:spin_vector} would no longer hold, since a spin perpendicular to the direction of motion requires that
%%
\begin{equation}
  s = (0,\,\hat{\vb{s}})\,, \hspace{5mm} \hat{\vb{s}} \cdot \hat{\vb{k}} = 0 \,,
\end{equation}
%%
where $\hat{\vb{k}}$ is the unit vector representing the direction of motion of the incoming lepton. Hence, the factor $E/m_{\ell}$ coming from Eq.~\eqref{eq:ch2:spin_vector} no longer appears to cancel the factor $m_{\ell}/E$ that arises in the cross-section difference Eq.~\eqref{eq:ch2:dsigma4}, such that this one turns out to vanish in the large energy limit $m_{\ell}/E \rightarrow 0$. \par
When talking about longitudinally or transversely polarisations one must specify the reference axis against which define the nucleon polarisation. It is customary in experiments to define the nucleon polarisation with respect to the leptom beam axis. In that regard, \textit{longitudinal} (\textit{transveral}) indicates the direction parallel (orthogonal) to the lepton beam and will be represented by the symbol $\Uparrow$ (\textit{transveral}). Having define the reference stage for the polarisation vectors, it is now straightforward giving a final expression from Eq.~\eqref{eq:cs_diff_fin_S} accounting for longitudinally or transversely polarised nucleons. In particular, for longitudinally polarised nucleons one has $\alpha=0$, yielding to
%%
\begin{equation}
  \frac{d^3 \, \sigma^{\uparrow, \Uparrow}}{dx \, dy \, d\varphi} - \frac{d^3 \, \sigma^{\uparrow, \Downarrow}}{dx \, dy \, d\varphi} = \frac{4 \alpha_{\T{em}}^2}{Q^2}  \left[ g_1(x,Q^2) \left( 2 - y - \frac{y^2 \gamma^2}{2} \right) - g_2(x,Q^2) y \gamma^2 \right] \,,
  \label{eq:cs_diff_l}
\end{equation}
%%
whereas for transversely polarised nucleons $\alpha=\pi/2$ and
%%
\begin{equation}
  \frac{d^3 \, \sigma^{\uparrow, \Rightarrow}}{dx \, dy \, d\varphi} - \frac{d^3 \, \sigma^{\uparrow, \Leftarrow}}{dx \, dy \, d\varphi} = \frac{4 \alpha_{\T{em}}^2}{Q^2} \gamma\sqrt{1 - y - \frac{y^2 \gamma^2}{4}}  \Bigl[ y g_1(x,Q^2) + 2 g_2(x,Q^2) \Bigr] \cos \phi\,.
  \label{eq:cs_diff_t}
\end{equation}
%%
In both \eqref{eq:cs_diff_l} the structure function $g_2$ is suppressed by the factor $\gamma$, Eq.~\eqref{eq:gamma}, whereas $g_1$ is not. From a phenomenological point of view, only $g_1$ is asymptotically relevant in the Bjorken limit $Q \rightarrow \infty$. In the other hand, the expression for the transversely polarisation is suppressed by an overall factor $\gamma$, even though both $g_1$ and $g_2$ equally contribute to the whole expression. For these reasons, experiments with longitudinally polarised targets are those of phenomenological interest, in that they allow to decouple $g_1$ from the rest of the expression Eq.~\eqref{eq:cs_diff_l}.\par
Finally, the unpolarised cross-section can be retrieved from Eq.~\eqref{eq:cross_section} by averaging over the spin of the incoming lepton $(s_{\ell})$ and of the nucleon $S$
%%
\begin{equation}
  \frac{d^3 \sigma}{dx dy d\varphi} = \frac{1}{2} \sum_{s_{\ell}} \frac{1}{2} \sum_{S} \frac{d^3 \sigma^{s_{\ell},S}}{dx dy d\varphi} = \frac{\alpha_{\T{em}}^2 y}{2 Q^4} L_{\mu \nu}^{(S)}W^{(S)\mu \nu}\,.
\end{equation}
%%
In the same way of the polarised case, the cross-section can be written in terms of the unpolarised structure functions $F_1$ and $F_2$. After integrating over the azimuthal angle $\varphi$, the expression for the cross-sections reads (Ref.~\href{https://pdg.lbl.gov/2019/reviews/rpp2019-rev-structure-functions.pdf}{PDG})
%%
\begin{equation}
  \frac{d^2 \sigma}{dx dy} = \frac{4 \pi \alpha_{\T{em}}^2}{xyQ^2} \left[ xy^2 F_1(x,Q^2) + (1-y + \frac{x^2y^2 M^2}{Q^2})F_{2}(x,Q^2)\right] \,,
  \label{eq:unp_cs}
\end{equation}
%%
where the contribution of order $M^2/Q^2$ that multiplies the structure function $F_2$ can be neglected in the Bjorken limit $M/Q \rightarrow 0$. In this limit, Eq.~\eqref{eq:unp_cs} can be written as
%%
\begin{equation}
  \frac{d^2 \sigma}{dx dy} = \frac{2 \pi \alpha_{\T{em}}^2}{xyQ^2} \left[ F_2(x,Q^2) Y_{+} - y^2 F_{L}(x,Q^2) \right]\,,
\end{equation} 
%%
where $Y_{+} = 1 + (1-y^2)$ and 
%%
\begin{equation}
  F_{L} = F_2 - 2xF_1 \,.
  \label{eq:Callan_Gross}
\end{equation}

%___________________________________________________________________________
\section{Phenomenology of DIS}
As previously introduced, deeply inelastic processes with different spin configurations of the target nucleon provide data from which gather information about the structure functions $g_1$ and $g_2$. However, the cross-section difference is a combination of these two functions, whose separation is achieved by a further approximate procedure. The phenomenological approach used hereafter will only focus on the polarised structure function $g_1$ since it is the one related to \textit{longitudinally polarised targets} that are used in experiments. Further details on the structure function $g_2$, which accounts for \textit{transversely polarised targets}, may be found in [\href{https://arxiv.org/abs/hep-ph/9501369}{arXiv:hep-ph/9501369}, \cite{leader_2001}, \cite{leader_predazzi_1996}].\par
The quantity that is actually measured in experiments with longitudinally or transversely polarised nucleon targets is a cross-section asymmetry defined as
%%
\begin{equation}
  A_{\parallel} = \frac{d\sigma^{\uparrow \Uparrow} - d\sigma^{\uparrow \Downarrow}}{d\sigma^{\uparrow \Uparrow} + d\sigma^{\uparrow \Downarrow}} \,, \hspace{5mm}  A_{\perp} = \frac{d\sigma^{\uparrow \Rightarrow} - d\sigma^{\uparrow \Leftarrow}}{d\sigma^{\uparrow \Rightarrow} + d\sigma^{\uparrow \Leftarrow}} \,,
  \label{eq:asymmetry}
\end{equation}
%%
where $d\sigma$ is a shorthand notation for $d^3\sigma/dxdyd\varphi$ and the denominator is twice the unpolarised cross-section. The asymmetries are usually related, via the optical theorem, to the imaginary part of the amplitude for the forward (off-shell) photo-absorption cross-section asymmetries $A_1$ and $A_2$ according to
%% 
\begin{align}
  A_{\parallel} = D(A_1 + \eta A_2)\,, \hspace{5mm} A_{\perp} = d(A_2 - \xi A_1)\,,
\end{align}
%%
where 
%%
\begin{equation}
  A_1 = \frac{\sigma_{1/2}^T - \sigma_{2/3}^T}{\sigma_{1/2}^T + \sigma_{2/3}^T} \,, \hspace{5mm} A_2 = \frac{\sigma_{1/2}^{TL}}{\sigma_{1/2}^T + \sigma_{2/3}^T}\,.
\end{equation}
%%
Here $\sigma_{1/2}^T$ and $\sigma_{3/2}^T$ represent the cross-sections for the forward virtual Compton scattering with a transversely polarised photon with total helicity of the photon-nucleon system equal to 1/2 and 2/3 respectively; $\sigma_{1/2}^{TL}$ represents the interference term between the longitudinal and transversal contributions to the photo-absorption cross-section (for more details, see Ref.~\cite{leader_predazzi_1996}). Inverting Eqs.~[\ref{eq:cs_diff_l}-\ref{eq:cs_diff_t}] yields to the expressions for $g_1$ and $g_2$ in terms of the cross-section asymmetries $A_{\parallel}$ and $A_{\perp}$
%%
\begin{align}
  & g_1(x,Q^2) = \frac{F_1(x,Q^2)}{(1+\gamma^2)(1 + \eta \zeta)} \left[ (1 + \gamma \zeta) \frac{A_{\parallel}}{D} - (\eta - \gamma) \frac{A_{\perp}}{d} \right] \,, \\
  & g_2(x,Q^2) = \frac{F_1 (x, Q^2)}{(1 + \gamma^2)(1 + \eta \zeta)} \left[ \left( \frac{\zeta}{\gamma} - 1 \right) \frac{A_{\parallel}}{D} + (\eta + \frac{1}{\gamma}) \frac{A_{\perp}}{d} \right] \,,
\end{align}
%%
where the kinematics factor have been defined as 
%%
\begin{align}
  & \varepsilon = \frac{4(1-y) - \gamma^2 y^2}{2 y^2 + 4(1-y) + \gamma^2 y^2} \,,\\
  & \zeta = \frac{\gamma(1 - y/2)}{1 + \gamma^2y/2} \,,\\
  & \eta = \frac{\varepsilon y \gamma}{1 - \varepsilon(1 - y)} \,,\\
  & D = \frac{1 - (1 - y)\varepsilon }{1 + \varepsilon R (x,Q^2)} \,,\\
  & d = D \frac{\sqrt{1 - y - \gamma^2 y^2 /4}}{1 - y/2}\,,
\end{align}
%%
and the unpolarised structure function ratio R can either be expressed by means of $F_1$ and $F_2$ or $F_2$ and $F_{L}$
%%
\begin{equation}
  R(x, Q^2) = \frac{F_2(x,Q^2)}{2 x F_1 (x, Q^2)} - 1 = \frac{F_{L}(x,Q^2)}{F_2 (x,Q^2) - F_{L}(x,Q^2)}\,.
\end{equation}
%%
It is also possible to express the structure functions in terms of $A_1$ and $A_2$ by using Eq.
%%
\begin{align}
  & g_1(x,Q^2) = \frac{F_1 (x,Q^2)}{1 + \gamma^2} \left[ A_1(x,Q^2) + \gamma A_2 (x,Q^2) \right] \,,
  \label{eq:g1_A12}\\
  & g_2(x,Q^2) = \frac{F_1 (x,Q^2)}{1 + \gamma^2} \left[ \frac{A_2}{\gamma} - A_1 \right] \,.
\end{align}
\par
In principle, separate measurements of $A_{\parallel}$ and $A_{\perp}$ are necessary to provide a direct determination of the structure functions $g_1$ and $g_2$, but in practice the majority of the experiments are performed with longitudinally polarised targets, hence only measuring $A_{\parallel}$. From Eq.~\eqref{eq:g1_A12} it is easy to see that, up to corrections of order $\mathcal{O}(M/Q)$, $g_1$ is completely determined by $A_1$, leading to the approximation
%%
\begin{equation}
  A_1 \approx \frac{A_{\parallel}}{D} \approx \frac{g_1}{F_1}\,.
  \label{eq:A1}
\end{equation}
%%
On the other hand, $g_2$ is determined by $A_{\perp}$ or $A_2$, even though it enters the analysis only as small corrections which can be neglected unless $Q^2$ is very small. 


\section{A look inside the hadronic tensor}
The hadronic tensor $W_{\mu \nu}$ introduced in the previous section contains information about the structure of the nucleon and describes the interaction between the virtual photon and the composite nucleon. It has been parametrised in terms of four structure functions which can be in principle measured by performing DIS experiments.\par
We are interested in the structure function $g_1$, in that it can be expressed in terms of longitudinally polarised quark and gluon distributions, as we shall see. Even though QCD is known to be \textit{asymptotically free}, the structure function does involve non-perturbative contributions, since the initial state (the nucleon) is not the fundamental degrees of freedom of the theory, but a compound state of quarks and gluons. Note that even in the Naive Parton Model such a picture holds, hence introducing a sort of unknown-structure dependence in the analysis. The depth by which this internal structure is probed, that is strongly depends on the theoretical model adopted to describe the nucleon and the interaction of its constituents, thus determining the way by which these distributions enter the expression for the structure function. We shall see that it is possible to give a \textit{factorised} expression for the structure function $g_1$, which allows for the separation of a hard, perturbative and process dependent from a low-energy, universal contribution. The former can be worked out by computing the diagrams in perturbative QCD, whereas the latter is given by the Parton Distribution Functions (PDFs), which parametrise the inner structure of the proton and cannot be computed perturbatively.\par
Before moving on the framework of QCD, in this section we shall first make use of the Partonic Model which gives an intuitive description, though not exhaustive, of nucleon structure and also provide the first historical approach to the puzzle of the nucleon structure. Then, the same results will be retrieved by the QCD at LO, while higher order corrections not only provide improvements prediction accuracy, but also introduce other contribution (such as the gluon and the sea quarks) that naturally arise in the framework of QCD and that were not taken into account by the Parton Model. 


\subsection{The hadronic tensor in the Naive Parton Model}
The Parton Model was initially formulated by Feynman [\href{https://journals.aps.org/prl/abstract/10.1103/PhysRevLett.23.1415}{Phys.Rev.Lett. 23 (1969) 1415}] and formalized by Bjorken and Paschos [\href{https://journals.aps.org/pr/abstract/10.1103/PhysRev.185.1975}{Phys.Rev. 185 (1969) 1975}] with the intention of providing a microscopic description of DIS before the advent of QCD. The idea relies on the infinite-momentum frame of reference for the proton, which coincides with the ceneter-of-mass frame in high-energy DIS experiments. When viewed from this frame, the proton results Lorentz-contracted into a thin shape in the direction of the motion, so that the incident lepton scatters instantaneously from the individual constituents. Moreover, time dilatation implies that the last interaction of the constituents occurred far away in time, so that it is possible to neglect the interactions that bind the constituents into a hadron. Based on this picture, the parton model asserts that the inclusive DIS can be approximated as an incoherent sum of lepton-constituent quark interactions. Furthermore, the incoming and outgoing partons can be considered as massless free particles, in so far the momentum transfer $Q$ is large. In the infinite-momentum frame the transversal momentum can be neglected, as it of order $M^2/Q$ (see Collins). Thus, partons transport a fractional momentum $\xi$ of the nucleon.\par
The parton model asserts that the cross-section can be computed as
%%
\begin{equation}
  \frac{d^2 \sigma}{dx dy} = \sum_{q} e_q^2 \int d\xi \;f_q (\xi) \frac{d \hat{\sigma}}{dy}\,,
  \label{eq:PM_CS}
\end{equation}
%%
where $d \hat{\sigma}$ is the partonic level cross-section for the elementary QED process $\ell \; q \rightarrow \ell' \; q$ and $f_q$ is the PDF, that is the probability density distribution of finding a parton of type $q$ with nulceon momentum fraction $\xi$. For polarised cross-section asymmetries, we should write
%%
\begin{equation}
  \frac{d \sigma^{\uparrow \Uparrow}}{dx dy} - \frac{d \sigma^{\uparrow \Downarrow}}{dx dy} = \sum_{q} e_q^2 \int d\xi \; \Delta f_q (\xi) \left[ \frac{d \Delta \sigma^{\uparrow \uparrow}}{dy} - \frac{d \Delta \sigma^{\uparrow \downarrow}}{dy} \right]
  \label{eq:PM_fact}
\end{equation}
%%
where $d \sigma^{\lambda_{\ell} \, \lambda_{q}}$ is the elementary cross-section with helicity states of the incoming lepton, $\lambda_{\ell}$, and the struck parton, $\lambda_{q}$. Here we have also introduced the helicity-dependent, or polarised PDFs, $\Delta f_{q}$, defined as the momentum densities of partons with spin aligned parallel or antiparallel to the longitudinally polarised parent nucleon:
%%
\begin{equation}
  \Delta f_{q} (\xi) = f_{q}^{\uparrow}(\xi) - f_{q}^{\downarrow}(\xi) \,.
\end{equation}
%%
The assertion Eq.~\eqref{eq:PM_CS} ralso affects the expression for the hadronic tensor. Instead of a factor $1/s$ enetering Eq.~\eqref{eq:ch2:dsigma3}, at the partonic level there is instead a factor $1/xs$, since the squared center-of-mass energy of the lepton-parton scattering is $(xP + q)^2 \approx 2 x \, P \cdot q$. Hence, the parton model expression for the structure function is 
%%
\begin{equation}
  W^{(A) \mu \nu}= \sum_{q} e_q^2\int \frac{d \xi}{\xi} \; \Delta f_{q} (\xi) \Delta \mathcal{C}^{\mu \nu}_{q} (x,q,s) \, \delta((p + q)^2) \,,
  \label{eq:had_tens_PM}
\end{equation}
%%
where the $\delta$-function has been introduced in order the implement the massh-shell-ness of the outgoing massless parton. 
Here $\Delta \mathcal{C}^{\mu \nu}_{q}$ represents the matrix element of the photon-parton interaction that can be computed at the lowest order in Quantum Electrodynamics (QED) (\red{mettere un diagramma con vertice di QED tra fotone e partone. In appendice eventualmente riportare il calcolo.}):
%%
\begin{equation}
  \Delta \mathcal{C}_q^{\mu \nu} (x,q,s) = - 2 m_q \varepsilon_{\mu \nu \alpha \beta} s^{\alpha} q^{\beta}\,.
\end{equation}
%%
Let us introduce two projectors
%%
\begin{align}
  & P_3^{\mu \nu} = \frac{(P \cdot q)^2}{b M^2 (q \cdot S)} \left[ (q \cdot S) S_{\lambda} + q_{\lambda}  \right] P_{\eta} \varepsilon^{\mu \nu \lambda \eta} 
  \\
  & P_4^{\mu \nu} = \frac{1}{b} \left\{ \left[ \frac{(P\cdot q)^2}{M^2} + 2 (P \cdot q) x \right]S_{\lambda} + (q \cdot S)q_{\lambda} \right\} P_{\eta} \varepsilon^{\alpha \beta \lambda \eta}\,,
\end{align}
%%
with 
%%
\begin{equation}
  b = -4M \left[ \frac{(P\cdot q)^2}{M^2}  + 2 (P \cdot q) x - (q \cdot S)^2\right]\,,
\end{equation}
%%
such that the structure functions can be projected out of the hadronic tensor, Eqs.~[\ref{eq:ch2:had_tens_final_symm}-\ref{eq:ch2:had_tens_final_anti}]
%%
\begin{align}
  &P_3^{\mu \nu} W_{\mu \nu} = g_2\,, \\
  &P_4^{\mu \nu} W_{\mu \nu} = g_1 + g_2 \,.
\end{align}
%%
Thus, the structure functions $g_1$ and $g_2$ can be projected out from Eq.~\eqref{eq:had_tens_PM} by means of a proper combination of the two projectors, yielding the important result
%%
\begin{align}
  &g_1(x) \hspace{1mm} = \hspace{1mm} \frac{1}{2} \sum_{q} e_q^2 \Delta f_{q} (x, S) \,,
  \label{eq:g1_PM}\\
  &g_2 (x) \hspace{1mm} = \hspace{1mm} 0\,.
  \label{eq:g2_PM}
\end{align}
%%
Alternatively, one can obtain these results by computing the partonic-level cross-sections appearing in Eq.~\eqref{eq:PM_fact}
%%
\begin{equation}
  \frac{d \Delta \sigma^{\uparrow \uparrow}}{dy} = \frac{4 \pi \alpha_{em}^2}{Q^2} \frac{1}{y} \,, \hspace{5mm} \frac{d \Delta \sigma^{\uparrow \downarrow}}{dy} = \frac{4 \pi \alpha_{em}^2}{Q^2} \frac{(1-y)^2}{y}\,,
\end{equation}
%%
that when inserted into Eq.~\eqref{eq:PM_CS} lead to the expression
%%
\begin{equation}
  \frac{d \sigma^{\uparrow \Uparrow}}{dx dy} - \frac{d \sigma^{\uparrow \Downarrow}}{dx dy} = \frac{4 \pi \alpha_{em}^2}{Q^2} \left[ \sum_{q} e_q^2 \Delta f_{q}(x) (2-y)\right] \,,
\end{equation}
%%
which can be compared to Eq.~\eqref{eq:cs_diff_l} once the latter is integrated over the azimuthal angle $\varphi$ and the terms $\mathcal{O}(\gamma^2)$ are neglected, so that one obtains the same results reported in Eqs.~[\ref{eq:g1_PM}-\ref{eq:g2_PM}]. Therefore, in the Naive Parton Model one has
%%
\begin{equation}
  \begin{split}
    g_1(x) & = \frac{1}{2} \left[ \frac{4}{9} \left( \Delta u(x) + \Delta \bar{u}(x) \right) + \frac{1}{9} \left(  \Delta d (x) + \Delta \bar{b}(x)  \right) + \frac{1}{9}\left(  \Delta d (x) + \Delta \bar{b}(x)  \right) \right] \\
    & = \frac{1}{9} \left[ \frac{3}{4} \Delta T_3 (x) + \frac{1}{4} \Delta T_8 (x) + \Delta \Sigma (x)  \right] \,,
  \end{split}
  \label{eq:g1_NPM_ev}
\end{equation}
%%
where I have neglected the contribution of heavy quarks. In the second line I have introduced linear combinations of quark densities with specific transformation properties under the symmetry group $SU(3)_{f}$
%%
\begin{align}
  & \Delta \Sigma (x) = \Delta u^{+} (x) + \Delta d^{+} (x) +\Delta s^{+} (x) \,, \\
  & \Delta T_3 (x) = \Delta u^{+} (x) - \Delta d^{+} (x) \,, \\
  & \Delta T_8 (x) = \Delta u^{+} (x) + \Delta d^{+} (x) - 2 \Delta s^{+} (x) \,,
\end{align}
%%
being 
%%
\begin{equation}
 \Delta q^{\pm}(x) = \Delta q(x) \pm \Delta \bar{q}(x) \,.
\end{equation}
%%
These linear combinations transform respectively as a flavour singlet, the third component of an isotopic spin triplet and the eight component of an $SU(3)_{f}$ octet.\par 
The results of the Naive Parton Model shown in this section have been obtained in a heuristic way, based on some intuitive, though powerful, assumptions. However, it is relevant to mention that a justification for such a picture can be given in the so-called \textit{impulse approximation}, when the intrinsic Fermi motion of quarks is neglected. For a detailed derivation within the framework of the Parton Model in the impulse approximation, see chapter 16.9 of Ref. \cite{leader_predazzi_1996}.\par
Using this approach, one can also allows for a transverse momentum of the partons, that is $p^{\mu} \hspace{3mm}  = \hspace{3mm} \left( E_1, p_x, p_y, \xi P_{z} \right)$. In doing so, the structure function $g_2$ assumes a non-zero value. Still, the framework of Parton Model does not give an unambiguous way to calculate $g_2$, providing different results that are generally incompatible with each other. One reason for that is that the structure function $g_2$ is extremely sensitive to the difference between the mass of the approximatively free-qurk $m_q$ and that of the bound quark $m$. In the special limit of longitudinal polarisation this difference does not make any relevant alteration to the analysis, yielding the results shown in Eqs.~[\ref{eq:g1_PM}-\ref{eq:g2_PM}]. In the other hand, this is no longer true when the transverse spin is relevant, motivating the high sensitivity on the mass difference for $g_2$. Furthermore, keeping $m \neq m_{q}$, the expression for the antisymmetric part of the hadronic tensor that one obtains is not gauge invariant ($q^{\mu} W_{\mu \nu}^{(A)} \neq 0$), eventually restored when a field theoretic approach is adopted [\href{https://arxiv.org/abs/hep-ph/9401264}{arXiv:hep-ph/9401264}].\par
Finally, the Wilson Operator Product Expansion (OPE) can be applied to the Fourier transform of the nucleon matrix element of the electromagnetic current that hadronic tensor $W^{\mu \nu}$, Eq.\eqref{eq:ch2:hadronic_tensor}. In this way, the odd moments of $g_{1,2}$ can be expressed in terms of hadronic matrix elements of certain operators multiplied by perturbatively evaluable coefficient functions. In particular, if the fields that enter the electromagnetic current are treated as free fields (\textit{i.e.} one adopts the naive parton model), the first moment of the structure function $g_1$, relevant in the longitudinally polarised case, can be expressed as\footnote{If one allows for a $Q^2$ dependence, then the coefficients that multiplies the $a_i$ receive calculable  perturbative QCD corrections.}
%%
\begin{equation}
  \Gamma_{1}^{p} = \int_{0}^{1} dx \, g_{1}^{p}(x) = \frac{1}{12} \left\{ a_3 + \frac{1}{\sqrt{3}} a_8 + \frac{4}{3} a_{0} \right\}\,,
  \label{eq:g1_m1}
\end{equation}
%%
where the $a_i$ are hadronic matrix elements of the octet of $SU(3)_{f}$ axial-vector currents $J^{i}_{5 \mu}$ ($i=1,\dots,8$) and the flavour singlet axial current $J^{0}_{5 \mu}$ taken between proton states of definite momentum and spin direction. Since the only relevant axial-vector in the present analysis is the spin-vector of the nucleon, $S_{\mu}$, the matrix elements are conventionally defined by
%%
\begin{align}
  & \bra{P,S} J_{5 \mu}^{i} \ket{P,S} \hspace{1mm} = \hspace{1mm} M a_{i} S_{\mu} \,,
  \\
  & \bra{P,S} J_{5 \mu}^{0} \ket{P,S} \hspace{1mm} = \hspace{1mm} 2 M a_{0} S_{\mu} \,,
\end{align}
%%
where the currents are
%%
\begin{align}
  & J_{5 \mu}^{i} \hspace{1mm} = \hspace{1mm} \bar{\vb*{\psi}} \gamma_{\mu} \gamma_{5} \left( \frac{\lambda_{i}}{2} \right) \vb*{\psi}\, ,\\
  & J_{5 \mu}^{0} \hspace{1mm} = \hspace{1mm} \bar{\vb*{\psi}} \gamma_{\mu} \gamma_{5} \vb*{\psi}\, ,
\end{align}
%%
where $\lambda_{i}$ are the Gell-Mann matrices and $\vb*{\psi}$ is the triplet of $SU(3)_{f}$ 
%%
\begin{equation}
  \vb*{\psi} = \left(\begin{matrix}
    \psi_{u} \\
    \psi_{d} \\
    \psi_{s} 
  \end{matrix}\right) \,.
\end{equation}
%%
It can be shown (see \textit{e.g.} Ref.\href{https://arxiv.org/abs/hep-ph/9401264}{arXiv:hep-ph/9401264} and references therein) that $a_{3,8}$ are related to the two constants, $F$ and $D$, measured in the hyperon $\beta$-decay
%% 
\begin{align}
  & a_3 = F + D \,, \\
  & a_8 = \frac{1}{\sqrt{3}} (3F - D)\,.
\end{align}
%%
Hence, measurements of $\Gamma_{1}^{p}$ can provide an estimation of the matrix element $a_0$ by simply inverting Eq.~\eqref{eq:g1_m1}
%%
\begin{equation}
  a_0 =  \frac{3}{4} \left\{ 12 \Gamma_{1}^p - a_3 - \frac{1}{\sqrt{3}} a_8 \right\}\,.
\end{equation}
%%
These relations become strongly relevant when, in the framework of the Naive Parton Model, the hadronic matrix elements are expressed in terms of the polarised parton distributions (see \textit{e.g.} Ref.\href{https://arxiv.org/abs/hep-ph/9401264}{arXiv:hep-ph/9401264}), providing
%%
\begin{align}
  & a_0 \equiv \Delta \Sigma = \int_{0}^{1} dx \, \Delta \Sigma(x) \,, 
  \label{eq:a0_NPM}
  \\
  & a_3 = \int_{0}^{1} dx \, \left[ \Delta u^{+}(x) - \Delta d^{+}(x) \right] \,,
  \label{eq:a3_PM}
  \\
  & a_8 = \frac{1}{\sqrt{3}}\int_{0}^{1} dx \, \left[ \Delta u^{+}(x) + \Delta d^{+}(x) - 2 \Delta s^{+}(x) \right] \,.
  \label{eq:a8_PM}
\end{align}
%%
\red{Nel capitolo "Theoretical constraints" ricordare che queste sum rules richiedono l'assunzione di $SU(3)_{f}$ come simmetria esatta.}In fact, by looking at the physical significance of $\Delta \Sigma (x)$, it follows that
%%
\begin{equation}
  a_0 = \Delta \Sigma \equiv 2 \left< S_{z}^{\T{quarks} + \T{antiquarks}} \right> \,.
  \label{eq:a0_Sz}
\end{equation}
%%
Since $\vb*{p}_{\perp}$, the orbital angular momentum carried by the quarks is perpendicular to $\vb*{P}$ and does not contribute to $J_{z}$. Since the gluons either were not taken into account or were considered unpolarised, the expectation for a proton with helicity $1/2$ was 
%%
\begin{equation}
  \left< S_{z}^{\T{quarks} + \T{antiquarks}} \right> = J_{z} = \frac{1}{2}
  \label{eq:J_z_NPM}
\end{equation}
%%
In the 80's the EMC collaboration measured the contribution of quarks and antiquarks to the total spin of the proton [\href{https://inspirehep.net/literature/280143}{EMC} ]using Eq.~\eqref{eq:a0_Sz}. Surprisingly, instead of the expected result Eq.~\eqref{eq:J_z_NPM}, they got
%%
\begin{equation}
  \left< S_{z}^{\T{quarks} + \T{antiquarks}} \right> = 0.03 \pm 0.06 \pm 0.09 \,,
  \label{eq:EMC_exp}
\end{equation}
%% 
opening the "spin crisis in the Parton Model" [\cite{Leader_Anselmino}]. However, as pointed in [\href{https://arxiv.org/abs/1604.00305v2}{arXiv:1604.00305}], the spin crisis was nothing but a consequence of the lack of the correct understanding of the deep and rich structure of the strong interaction, which will be taken into account by QCD. The following section will provide a summary of modifications and improvements introduced by QCD.

%___________________________________________________________________________
\subsection{The Field Theoretic Model}
\label{sec:field_theoretic}

The heuristic approach introduced by the Naive Parton Model provides effective results, laying the groundwork for QCD. However, the calculation do not lead to coherent predictions when compared to experimental data. The "something" of QCD as the theory of strong interactions provides a more accurate description in highly accordance the experiments. In this framework, the accuracy of the predictions increases as the perturbative order of the Feynman diagram is increased. As previously discussed, at Born level the interaction is represented by the tree-level scattering of a quark (or antiquark) off the virtual photon $\gamma^{*}$, whose Feynman diagram is reported in Fig.~. In this case one tha Naive Parton Model expectations are recovered. At higher order QCD introduces several new contributions, which can be computed perturbatively. For instance, at $\mathcal{O}(\alpha_{s})$, the Feynman diagrams include the emission of a gluon, the one-loop corrections, and the process initiated by a gluon which then splits into a quark-antiquark pair, the so-called photon-gluon fusion (PGF) process.\par
QCD interactions affect the predictions by two main features
\begin{enumerate}
  \item They introduce a perturbative calculable logarithmic $Q^2$ dependence in the parton densities
  \item they naturally allow for the polarization of the gluons in the nucleon the gluon contributing to $g_1$.  
\end{enumerate}
\red{Questa parte la metterei dopo. Il punto è che la fattorizzazione vale solo al leading twist, ovvero solo quando viene considerato il diagramma in figura. In quel caso si applica la fattorizzazione.}
\\
In the field theoretic model, the leading diagram in which only quarks of four-momenta $p$ from the hadron are involved is shown in Fig.~\ref{fig:DIS_FT}. The interaction is split into 

%% Field theoretic DIS figure
\begin{figure}[h]
  \centering
  \includegraphics[width=0.5\textwidth]{DIS_FT.pdf} 
  \caption{Handbag diagram for DIS in the field theoretic approach.}
  \label{fig:DIS_FT}
\end{figure}
%%

\subsubsection*{Perturbative corrections and the evolution equation}
The inclusion of gluonic corrections to the Born level, such as those represented in Fig.~, forces dealing with infinite contributions. These divergent terms arise from the so-called \textit{mass} or \textit{collinear} singularities that occur because of the effective masslessness of the quarks and are removed by the \textit{factorisation} process. The \textit{factorisation theorem} 
\footnote{One should talk about 'factorisation theorems' since every process has its own theorem. However, since we are dealing with DIS, we can restrict us to the single theorem concerned in this process.}
[\href{https://arxiv.org/abs/hep-ph/0409313}{https://arxiv.org/abs/hep-ph/0409313}] allows for a separation of the interaction into a 'hard' perturbative part $H^{\mu \nu}_{\alpha \beta}$ (the top 'blob') and a 'soft' non-perturbative part $\Phi_{\alpha \beta}$ (the bottom 'blob'). Schematically, the computation of the hard part gives origin to terms of the form $\alpha_{s} \ln Q^2/m_q^2$, which are split as follows
%-------------------------------
\footnote{Here the divergent contribution is give by the mass term inside the logarithmic. However, this term arises only when computation of the hard part is worked out keeping the non-zero mass for the quark. Otherwise, the singularity arises as infrared logarithmic divergence of the type $\alpha_{s} \ln Q^2/\mathcal{K}^2$, where $\mathcal{K}$ is a small cut-off in momentum space.}
%-------------------------------
%%
\begin{equation}
  \alpha_{s} \ln \frac{Q^2}{m_q^2} = \alpha_{s} \ln \frac{Q^2}{\mu^2} + \alpha_{s} \ln \frac{\mu^2}{m_{q}} \,.
\end{equation}
%%
The first term on the r.h.s. is absorbed into the hard part, whereas the other one is absorbed into the soft part (\textit{i.e.} the PDF), which in any case has to be parametrized and studied experimentally. This procedure introduces a \textit{factorisation scale} $\mu^2$, that is the scale at which the separation is made. As a consequence, PDFs acquire a scale-dependence, whereas the remaining finite part incorporates $Q^2$-dependent terms which correct the expression for the structure functions, thus breaking the Bjorken scaling. \par
If all orders in perturbation theory were considered, the physical observables would not depend on this scale, thus making the choice of its values completely arbitrary. However, it turns out that predictions are truncated at some fixed order in perturbation theory, and choice of the values could make some difference. Usually, the optimal choice is $\mu^2 = Q^2$, so that the PDFs will now depend on $x$ and $Q^2$
%%
\begin{equation}
  \Delta f(x)  \hspace{2mm} \longrightarrow \hspace{2mm} \Delta f(x,Q^2) \,.
  \label{eq:Q2_pdf}
\end{equation}
%%
If one keeps only the leading logarithmic (LL) terms proportional to $\alpha_{s} \ln Q^2/\mu^2$, one recovers the Naive Parton Model expressions for the polarised structure functions, Eq.~\eqref{eq:g1_PM}, provided the replacement reported above.\par
The variation with $Q^2$ is controlled by the Gribov, Lipatov, Altarelli and Parisi evolution equations [\href{https://www.sciencedirect.com/science/article/abs/pii/0550321377903844?via%3Dihub}{Nucl.Phys. \textbf{B126} (1977) 298.}\red{manca quella di Gribov and Lipatov}], a set of $(2 n_f + 1)$ coupled integro-differential equations. For the polarised densities, these equation read
%%
%\begin{equation}
%  \begin{split}
%    \frac{\partial}{\partial \ln Q^2} \Delta q_{i} (x,Q^2) = \frac{\alpha_{s}(Q^2)}{4 \pi} \int_{x}^{1} \frac{dy}{y} &\left\{\sum_{k}^{n_f} \left[ \Delta P_{q_i q_k} \left( \frac{x}{y} \right) \Delta q_k(y,Q^2) + \Delta P_{q_i \bar{q}_k} \left(\frac{x}{y}\right) \Delta \bar{q}_k(y,Q^2) \right]\right. \\
%    & \left. + \Delta P_{q_i g} \left(\frac{x}{y}\right) \Delta g (y, Q^2) \right\}
%  \end{split}
%\end{equation}
%%
%\begin{equation}
%  \begin{split}
%    \frac{\partial}{\partial \ln Q^2} \Delta g_{i} (x,Q^2) = \frac{\alpha_{s}(Q^2)}{4 \pi} \int_{x}^{1} \frac{dy}{y} &\left\{\sum_{k=1}^{n_f} \left[ \Delta P_{g q_k} \left( \frac{x}{y} \right) \Delta q_k(y,Q^2) + \Delta P_{g \bar{q}_k} \left(\frac{x}{y}\right) \Delta \bar{q}_k(y,Q^2) \right]\right. \\
%    & \left. + \Delta P_{g g} \left(\frac{x}{y}\right) \Delta g (y, Q^2) \right\}
%  \end{split}
%\end{equation}
%%
\begin{equation}
  \frac{\partial}{\partial \ln Q^2} 
  \left(\begin{matrix}
    \Delta q_i \\
    \Delta g \\
    \Delta \bar{q}_i
  \end{matrix} \right) (x,Q^2) = \frac{\alpha_{s}(Q^2)}{4 \pi}  \sum_{k,l}
  \left(\begin{matrix}
    \Delta P_{q_i q_k} && \Delta P_{q_i g} && \Delta P_{q_i \bar{q}_l} \\
    \Delta P_{g q_k} && \Delta P_{g g} && \Delta P_{g \bar{q}_l} \\
    \Delta P_{\bar{q}_i q_k} && \Delta P_{\bar{q}_i g} && \Delta P_{\bar{q}_i \bar{q}_l} \\
  \end{matrix}\right) \otimes 
  \left(\begin{matrix}
    \Delta q_i \\
    \Delta g \\
    \Delta \bar{q}_i
  \end{matrix} \right) (x,Q^2) \,,
  \label{eq:DGLAP_coupled}
\end{equation}
%%
where $k,l$ run over the quark flavours ($k,l = u,\, d\, s,\, \dots$) and $\otimes$ is the shorthand notation for the convolution product with respect to x
%%
\begin{equation}
  f \otimes g = \int_{x}^{1} \frac{dy}{y} f \left(\frac{x}{y} \right) g(y) \,.
  \label{eq:def_conv}
\end{equation}
%%
The $\Delta P$ are the polarised \textit{splitting functions} and can be expanded in powers of the strong coupling $\alpha_s$
%%
\begin{equation}
  \Delta P = \sum_{n=0}^{\infty} \left( \frac{\alpha_s}{4\pi} \right)^{n} \Delta P^{(n)}(x)\,.
\end{equation}
%%
The evolution equations can maximally decoupled from each other if one exploits charge conjugation invariance and $SU(n_f)$ flavour symmetry, which impose 
%%
\begin{equation}
  \begin{split}
    & \Delta P_{q_i q_j} = \Delta P_{\bar{q}_i \bar{q}_j} = \delta_{ij} P_{qq}^{V} + P_{qq}^{S} \,,\\
    & \Delta P_{\bar{q}_i q_j} = \Delta P_{q_i \bar{q}_j} = \delta_{ij} P_{\bar{q}q}^{V} + P_{\bar{q}q}^{S} \,,\\
    & \Delta P_{q_i g} = \Delta P_{\bar{q}_i g} = \Delta P_{qg} \,, \\
    & \Delta P_{g q_i} = \Delta P_{g \bar{q}_i} = \Delta P_{gq}
  \end{split}
\end{equation}
%%
Inserting these relations into Eq.~\eqref{eq:DGLAP_coupled} and using
%%
\begin{equation}
  \Delta q_{i}^{\pm} = \Delta q_{i} \pm \Delta \bar{q}_{i} \,,
\end{equation}
%%
the evolution equations read 
%%
\begin{equation}
  \begin{split}
    \frac{\partial}{\partial \ln Q^2} 
    \left(\begin{matrix}
      \Delta q_i^{+} \\
      \Delta g \\
      \Delta \bar{q}_i^{-}
    \end{matrix} \right) & = \frac{\alpha_{s}(Q^2)}{4 \pi} 
    \left(\begin{matrix}
      (\Delta P_{qq}^{V} + \Delta P_{q\bar{q}}^{V}) && 2 \Delta P_{qg} && 0 \\
      \Delta P_{gq} && \Delta P_{gg} && 0 \\
      0 && 0 && (\Delta P_{qq}^{V} - \Delta P_{q\bar{q}}^{V})
    \end{matrix} \right) \otimes
    \left(\begin{matrix}
      \Delta q_i^{+} \\
      \Delta g \\
      \Delta \bar{q}_i^{-}
    \end{matrix} \right) 
    \\[10pt]
    & + \frac{\alpha_{s}(Q^2)}{4 \pi}
    \left(\begin{matrix}
      (\Delta P_{qq}^{S} + \Delta P_{q \bar{q}}^{S}) && 0 && 0 \\
      0 && 0 && 0 \\
      0 && 0 && (\Delta P_{qq}^{S} - \Delta P_{q\bar{q}}^{S})
    \end{matrix} \right) 
    \otimes
    \left(\begin{matrix}
      \sum_{k} \Delta q_k^{+} \\
      \Delta g \\
      \sum_{k} \Delta \bar{q}_k^{-}
    \end{matrix} \right)
  \end{split}
\end{equation}
%%
where the dependence on the kinematic has been momentarily omitted. Now the equations are semi-diagonalised, being the third equation completely decoupled from the rest of the system. It is convenient to express the system using the following definitions (here the sum runs over quarks and antiquarks)
%%
\begin{equation}
  \begin{split}
     \Delta \Sigma \equiv \sum_{k=1}^{n_{f}} \Delta q_{k}^{+}  \hspace{10mm} & \T{singlet distribution}\, ,\\
     \Delta V \equiv \sum_{k=1}^{n_{f}} \Delta q_{k}^{-}  \hspace{10mm} & \T{valence distribution}\,, \\
     \Delta q_{NS,is}^{\pm} \equiv \Delta q_{i}^{\pm} - \Delta q_{j}^{\pm} \hspace{10mm} & \T{non-singlet distribution} \,,
  \end{split}
  \label{eq:evb_dist}
\end{equation}
%%
for the polarised parton distributions and
%%
\begin{equation}
  \begin{split}
    & \Delta P^{\pm} \equiv \Delta P_{qq}^{V} \pm \Delta P_{q \bar{q}}^{V} \,,\\
    & \Delta P_{qq} \equiv \Delta P^{+} + n_{f} (\Delta P_{qq}^{S} + \Delta P_{q\bar{q}}^{S} )  \,,\\
    & \Delta P^{V} \equiv \Delta P^{-} + n_{f} (\Delta P_{qq}^{S} - \Delta P_{q \bar{q}}^{S}) \,,
  \end{split}
\end{equation}
%%
for the polarised splitting functions. Hence, the evolution equation can be expressed as follows
%%
\begin{equation}
  \begin{split}
    & \frac{\partial}{\partial Q^2} \Delta g  = \frac{\alpha_{s}(Q^2)}{4 \pi} \Bigl[ \Delta P_{gg} \otimes g + \Delta P_{gq} \otimes \Delta \Sigma \Bigr] \,,\\
    & \frac{\partial}{\partial Q^2} \Delta q^{+}_{i} = \frac{\alpha_{s}(Q^2)}{4 \pi} \Bigl[ \Delta P^{+} \otimes \Delta q_{i}^{+} + \frac{1}{n_f} (\Delta P_{qq} - \Delta P^{+}) \otimes \Delta \Sigma + 2 \Delta P_{qg} \otimes \Delta g \Bigr] \,,\\
    & \frac{\partial}{\partial Q^2} \Delta q^{-}_{i} = \frac{\alpha_{s}(Q^2)}{4 \pi} \Bigl[ \Delta P^{-} \otimes \Delta q_{i}^{-} + \frac{1}{n_f} (\Delta P^{V} - \Delta P^{-}) \otimes \Delta V \Bigr] \,,\\
  \end{split}
  \label{eq:evb_1}
\end{equation}
%%
At this point, it is customary to define the so-called \textit{evolution basis}, a specific set of linear combinations of distribution that maximally decouple the evolution equations. In particular, by looking at the Eqs.~\eqref{eq:evb_1} and using the definitions in Eqs.~\eqref{eq:evb_dist}, it is straightforward to verify that the evolution equations completely decouple for the non-singlet and valence distribution
%%
\begin{equation}
  \begin{split}
    & \frac{\partial}{\partial \ln Q^2}\Delta q_{\T{NS}ij}^{\pm} (x,Q^2) = \frac{\alpha_s(Q^2)}{4\pi} \Delta P^{\pm} \otimes \Delta q_{{\T{NS}ij}}^{\pm} (x,Q^2) \,, \\
    & \frac{\partial}{\partial \ln Q^2}\Delta V (x,Q^2) = \frac{\alpha_s(Q^2)}{4\pi} \Delta P^{V} \otimes \Delta V (x,Q^2)\,,
  \end{split}
\end{equation}
%% 
whereas the singlet and the gluon distributions remain coupled 
%%
\begin{equation}
  \frac{\partial}{\partial \ln Q^2} 
  \left(\begin{matrix}
    \Delta g (x,Q^2)\\
    \Delta \Sigma (x,Q^2)
  \end{matrix} \right) 
  = \frac{\alpha_{s}(Q^2)}{4 \pi} \otimes
  \left(\begin{matrix}
    \Delta P_{gg} && \Delta P_{gq} \\
    \Delta P_{qq} && \Delta 2n_f P_{qg}
  \end{matrix} \right)
  \otimes 
  \left(\begin{matrix}
    \Delta g (x,Q^2) \\
    \Delta \Sigma (x,Q^2)
  \end{matrix} \right) \,,
\end{equation}
%%
and we have restored the dependence on the kinematics $(x,Q^2)$. The polarised splitting functions have been computed at LO in [\href{https://www.sciencedirect.com/science/article/abs/pii/0550321377903844?via%3Dihub}{Nucl.Phys. B126 (1977) 298.}], then extended at NLO in [\href{http://dx.doi.org/10.1007/s002880050138}{Z.Phys. C70 (1996) 637}], [\href{http://dx.doi.org/10.1103/PhysRevD.54.2023}{Phys.Rev. D54 (1996) 2023,}]; only recently splitting functions have been computed at NNLO in [\href{https://inspirehep.net/literature/1317880}{Nucl.Phys.B 889 (2014) 351-400}].\par
The computation of the structure functions must take into account the scale at which the experiment is performed, introducing additional complexity to analysis. In Section 3 we will address the numerical implementation of the evolution equation, allowing for the evolution of the PDFs in the computation of the predictions.\par
\red{What's missing:}
\begin{enumerate}
  \item Physical interpretation of the DGLAP equation\\
  \item Paragraph in page 20 in which further details on the physical interpretation of the splitting functions are given
  \item When I talked about the divergent terms that is absorbed into the soft part, I should mention that because of that the regular part acquires perturbative $Q^2$-dependent contributions that affect the fucntional form of the structure function. Then I would conclude that the expression for the structure function is given in the second subsection
\end{enumerate}
Finally, I would like to conclude this section with complementary observations about the factorisation theorem. In principle, there could be other diagrams contributing to the process in the field theoretic approach. When factorisation theorem only refers to Fig.~\ref{fig:DIS_FT}, which represents the leading regions for DIS when the light-cone gauge, $A^{+}=0$, is adopted. All the contributions coming from this diagram are called 'leading twist'. All the diagrams other than that shown in Fig.~\ref{fig:DIS_FT} have an overall suppression of order $(1/Q)^{n}$ and provide the so-called 'higher twist' corrections. Therefore, the complete expression for an observable such as the structure functions should read
\begin{equation}
  F(x, Q^2) = F(x, Q^2)^{LT} + \frac{F^{(1)}(x,Q^2)}{Q^2} + \dots \,.
\end{equation}
For instance, target mass corrections (TMCs), which originate from nonzero values of hadron masses, provide subleading corrections of the type $\mathcal{O}(M/Q)^2$. In this thesis we will neglect this higher-twist corrections, given that we apply a cut in the kinematic region covered by the data to ensure the reliability of the factorisation theorem without higher order terms. 

\subsubsection*{The gluon contribution to $g_1$}
Another important consequence of the QCD analysis is the rise of the contribution from polarised gluons to the structure function $g_1$. Indeed, the leading-twist expression becomes 
%%
\begin{equation}
  g_1(x,Q^2) = \frac{1}{2} \sum_{i=1}^{n_f} e_{q}^2 \Biggl\{ \Delta \mathcal{C}_{q} \otimes \Bigl[ \Delta q_i (x,Q^2) + \Delta \bar{q}_i (x,Q^2) \Bigr] + \Delta \mathcal{C}_{g} \otimes \Delta g (x,Q^2)\Biggr\} \,,
  \label{eq:g1_QFT}
\end{equation}
%%
where the sum is over the flavours of quarks and antiquarks, $\otimes$ denotes the usual convolution in Eq.~\eqref{eq:def_conv}, and finally $\Delta \mathcal{C}_{q}$ and $\Delta \mathcal{C}_{g}$ are the \textit{coefficient functions} which are related to the hard photon-quark or photon-gluon cross-sections. Coefficient functions are perturbative quantities and can be expanded in power series of $\alpha_s$
%%
\begin{equation}
  \Delta \mathcal{C} \left( y, \alpha_s \right) = \Delta \mathcal{C}^{(0)}_{p} (y) + \frac{\alpha_s}{4\pi} \Delta \mathcal{C}^{(1)}_{p} (y) + \mathcal{O}(\alpha_s^2)\,,
\end{equation}
%% 
where $p=q,g$ and
%%
\begin{equation}
  \left\{ \hspace{-3mm}
  \begin{array}{cl}
    &\Delta \mathcal{C}_{q}^{(0)} (y) = \delta(1-y)\,,\\[10pt]
    &\Delta \mathcal{C}_{g}^{(0)} (y) = 0 \,. \\
  \end{array}
  \right.
\end{equation}
%%
At lowest order the polarised gluon distributions do not contribute to the structure function, hence recovering the Naive Parton Model expectation, Eq.~\eqref{eq:g1_PM}.\par
Now that we have introduced the polarised gluon distribution, we can revise the unexpected result, Eq.~\eqref{eq:EMC_exp}, that questioned the reliability of the NPM. In the NPM the axial current $j_{5\mu}^{0}$ can be regarded as the spin quark operator, since we showed before that it is related to the first moment of the singlet quark distribution. The key point to understand is that the axial current $j_{5 \mu}^{0}$ is conserved only in a massless and \textit{free-field} theory. However, when one allows for interaction among quarks, the current is no longer conserved, even when the masses of partons are neglected. In principle, one would not be worried about that, given that the conserved quantity is total angular momentum $J_z$ and not $S_z$, so that the physical interpretation of $j_{5\mu}^{0}$ as the spin operator of quark remains valid. However, it turns out that the current is not conserved because of an anomalous contribution coming from the triangle diagram that arises from the axial gluon current(\red{DIAGRAM}). This suggests the possible presence of an additional contribution to the proton spin which has not been considered in the expectation value for $j_{5\mu}^{0}$ in the NPM, thus questioning the physical interpretation of the operator $j_{5\mu}^{0}$ and of the singlet quark first moment, which must be revisited within an interactive theory. (\red{se poi introduciamo anche una dipendenza da $Q^2$ in $\Delta \Sigma$ allora abbiamo qualcosa che dipende dalla scala, e che quindi non può rappresentare lo spin dei quark (perde di interpretazione fisica.) Questa è diretta conseguenza dell'anomalo contributo portato dal gluone nell'accoppiamento con il fotone }).\par
It can be shown [30, 139 and 140 of PhD thesis] that the gluon contribution to the structure function $g_1$ modifies the expression modifies the expectation value of the singlet axial current by a term that reads
%%
\begin{equation}
  a_{0}^{\T{gluons}}(Q^2) = - n_{f} \frac{\alpha_s(Q^2)}{2\pi} \int_{0}^{1} dx \, \Delta g (x, Q^2) \,,
\end{equation}
%%
where $\Delta g (x,Q^2)$ is the total helicity carried by the gluons and $n_{f}$ indicated the number of active flavours beyond that effectively participate in the analysis (in this case, $n_f = 3$ since we are considering only the lightest quarks $u,d$ and $s$, whereas heavy flavours are assumed not to participate). Therefore, the expression that accounts for the anomaly analogous to the NPM expectation should read
%%
\begin{equation}
  a_{0} (Q^2) = \Delta \Sigma (Q^2) - 3 \frac{\alpha_{s}(Q^2)}{2\pi} \Delta g (Q^2) \,,
  \label{eq:a0_gluon}
\end{equation}
%%
where we have considered $n_f = 3$ active flavours and $\Delta \Sigma(Q^2)$, $\Delta g(Q^2)$ represent the first moment of the singlet and gluon distributions, respectively. What the EMC experiment actually measured was the total singlet axial charge $a_0$. The l.h.s. of this expression has the fundamental implication that the small measured value of $a_0$ can now be explained as a cancellation between the two terms, reconciling the discrepancy between the EMC result and the theoretical expectation. Now, as a general rule, the NPM can be derived from perturbative QCD in the limit $Q^2 \rightarrow \infty$, since the running coupling $\alpha_s$ vanishes and the free-theory is recovered. However, an exception to this rule is the gluon contribution in Eq.~\eqref{eq:a0_gluon} Even though it may be seen as a perturbative correction to $a_0$, given that it is multiplied be $\alpha_s$, the gluon contribution does not decouple in the limit $Q \rightarrow \infty$. Indeed, it can be shown either from the spin-dependent DGLAP evolution equations [ALT77] or from the anomalous dimensions of the operators involved [\href{https://arxiv.org/abs/hep-ph/9501369}{arXiv:hep-ph/9501369}], that the logarithmic decrease of $\alpha_s$ as $Q^2 \rightarrow \infty$ is just compensated by the increase in gluon helicity distribution. This is a direct consequence of the contribution induced by the axial anomaly in the triangle diagram. Thus, the initial physical interpretation of $\Delta \Sigma (Q^2)$ relied on the conservation of the singlet axial current $j_{5\mu}^{0}$. There is no reason for the first moment of the singlet distribution to coincide with the value of the constituent quark model. This inconsistency is also made more clear by the fact that the first moment of the singlet distribution, being related to a not conserved quantity (the singlet axial current), acquires a scale dependence, that reifies in a $Q^2$ dependence. It is then clear that an observable quantity such as the spin of the quarks cannot depend on an in-principle arbitrary quantity such as the scale, making the physical significance of the singlet quark distribution $\Delta \Sigma(x,Q^2)$ a matter of a careful definition, as we shall see later.\par
Finally, we can try to give a rough estimate of the gluon helicity. The expected spin contribution carried by quarks is $\Delta \Sigma \simeq (0.6  \div 0.7)$. To accomplish the huge cancellation enclosed in the singlet axial charge, the gluon contribution at $Q^2 \simeq 10 \, \T{GeV}^2$ is should be $\Delta g \simeq 4$. Despite the large value, it cannot be ruled out in that the QCD evolution equations increase indefinitely the value of $\Delta g$ with $Q^2$. Furthermore, as we shall see later, angular momenta must satisfy a helicity sum rule which reads
%%
\begin{equation}
  J_{z} = S_{z}^{q} + S_{z}^{g} + L_z = \frac{1}{2} \,, 
  \label{eq:J_z_sumrule}
\end{equation} 
%%
where $L_z$ represents the total angular momentum of all partons. Hence, in order for Eq.~\eqref{eq:J_z_sumrule} to be fulfilled, the growing value of $S_{z}^{q}$ has to be compensated by an analogous growth in the magnitude of $L_{z}$. Since a measurement of the orbital momentum of partons is hardly achievable, the only possibility to estimate this contribution is by determining both quarks and gluon contributions from experimental data, providing a reliable accuracy. The goal of this thesis is to provide such estimates, by scrutinizing both the available experimental data and the methodology.

\subsection{Scheme dependence}
Given that we want to achieve higher accuracies by higher order perturbative corrections, we must deal with infinities. These are regulated by renormalization, which introduces a dependence on the chosen scheme. Despite the fact that the choice is completely arbitrary, it may strongly affect the results and compromise the physical interpretation of certain quantities, as we showed in the section before. Her, we just outline the direct consequences of different schemes, without providing any details. A pedagogical introduction to the scheme dependence may be found in Ref.~\cite{leader_2001} and reference therein. 
\begin{enumerate}
  \item The most frequently used renormalization scheme is the $\overline{\T{MS}}$-scheme, in which the gluon coefficient functions vanish. As a result, there is no gluon contribution to the first moment of the structure function $g_1$. The first moment of the singlet quark distribution, $\Delta \Sigma (Q^2)$, varies with $Q^2$ and any interpretation as the total spin carried by quarks can be used. Hence, the singlet axial charge 
  %%
  \begin{equation}
    a_0 (Q^2) = \int_{0}^{1} dx \,  \left.\Delta \Sigma(x,Q^2) \right|_{\overline{\T{MS}}} \,,
    \label{eq:MSB_scheme}
  \end{equation}
  %%
  cannot be compared with constituent quark result. Finally, the first moment of the non-singlet triplet and octet distribution,
  %%
  \begin{equation}
    a_3 = \int_{0}^{1} dx \, \Delta T_3 (x,Q^2) \,, \hspace{10mm} a_8 = \int_{0}^{1} dx \, \Delta T_8 (x,Q^2) \,,
  \end{equation}
  %% 
  do not have any $Q^2$-dependence. These latter will be adopted as theoretical assumption to further constraint the PDF determination, as we shall we in the next chapter.
  
  \item The other renormalization scheme is due to Adler and Bardeen (AB), and is defined such that the first moment of the singlet distribution is independent of $Q^2$, restoring the physical interpretation as the total spin carried by quarks. The gluon distribution is the same as in the $\overline{\T{MS}}$-scheme, but now it contributes to the first moment of the gluon of the structure function $g_1$
  %%
  \begin{equation}
    a_{0}(Q^2) = \int_{0}^{1} dx \, \left. \Delta \Sigma (x,Q^2) \right|_{\T{AB}} - n_f \frac{\alpha_s(Q^2)}{2\pi} \int_{0}^{1} dx \, \Delta g(x,Q^2) \,.
    \label{eq:AB_scheme}
  \end{equation}
\end{enumerate}
The two approaches are related by comparing Eq.~\eqref{eq:MSB_scheme} with Eq.~\eqref{eq:AB_scheme}
%%
\begin{equation}
  \int_{0}^{1} dx \, \left. \Delta \Sigma (x,Q^2) \right|_{\T{AB}} = \int_{0}^{1} dx \, \left. \Delta \Sigma (x,Q^2) \right|_{\overline{\T{MS}}} + n_f \frac{\alpha_s(Q^2)}{2\pi}  \int_0^1 dx \, \Delta g(x,Q^2)\,,
  \label{eq:AB-MS}
\end{equation}
%%
It must be noted that the scheme dependence, which highly affects the definition of $\Delta \Sigma (x,Q^2)$, persists even in the limit $Q^2 \rightarrow \infty$. In fact, the two definitions that appear in Eq.~\eqref{eq:AB-MS} differ by a term which is not asymptotically suppressed in this limit, as discussed above. Hence, the definition of the singlet quark first moment is therefore maximally ambiguous.

\section{Semi-inclusive polarised DIS}
In inclusive DIS, only a precise combination of distributions, Eq.~\eqref{eq:g1_NPM_ev}, is probed by data, resulting in a low constraint on the single flavour distributions. Semi-inclusive processes aim to achieve a higher flavour separation if compared with inclusive experiments. In particular, we will consider data from semi-inclusive deeply inelastic processes, which are the same as inclusive DIS, except that a certain particle (typically a pion or a kaon) is detected in the final state. The process may be sketched as
%%
\begin{equation}
  \ell + N \longrightarrow \ell' + h +  X \,,
    \label{eq:SIDIS}
\end{equation}
%%
where $h$ represents the detected hadron. As with DIS, the squared amplitude factorises into a leptonic and a hadronic part. While the former does not get modified, the latter now reads [Collins]
%%
\begin{equation}
  W^{\mu \nu} (q,P) = \frac{1}{4\pi} \sumint_{X} d^4 z \, e^{i q \cdot z} \, \bra{0} j^{\mu}(z/2) \ket{P,X,\T{out}} \, \bra{P,X,\T{out}} j^{\nu}(-z/2) \ket{0}\,.
\end{equation}
%%
Contrary to the DIS, now we select a hadron in the final state, so that we cannot eliminate the sum $\sumint_{X}$ as we did for DIS. This dependence on the final state hadron reifies in the introduction of an additional non-perturbative quantity - the fragmentation function (FF). A detailed analysis on this new soft contribution is beyond the scope of this Thesis, and we refer to Ref.~[\href{https://arxiv.org/abs/1607.02521v2}{arXiv:1607.02521 }] for a formal introduction to FFs. For the present work, it is necessary to note that FFs provide the analogue for PDFs for the final hadronization state, that is how quarks and gluons combine into a colour-neutral particle such as hadrons. The FF is denoted by $D^{h/i}(z)$ and describe the fragmentation of an unpolarised parton of type $i$ into an unpolarised hadron of type $h$. The variable $z$ is the counterpart of the Bjorken variable $x$ and, in the intuitive picture provided by the PM, it represents the fraction of the parton momentum carried by the detected hadron. Again, in the simpified picture of the PM, the fraction momentum $z$ only refers to the longitudinal component along the direction of motion of the parton. In the PM, the quantity $D^{h/i}(z)dz$ can be interpreted as the number of hadrons $h$ originating from the parton $i$ with momentum faction in $[z, z+dz]$, although such a view is strongly modified when accounting for higher-order QCD corrections.\par
\begin{figure}
  \centering
  \includegraphics[width=0.5\textwidth]{SIDIS_2.pdf} 
  \caption{Parton model for SIDIS.}
  \label{fig:SIDIS_PM}
\end{figure}
The kinematic variables are the same of the DIS, Eqs.~[\ref{eq:ch2:nu}-\ref{eq:ch2:W2}], with the additional variable
%%
\begin{equation}
  z = \frac{P \cdot p_h}{P \cdot q} \,,
\end{equation}
%%
where $p_h$ is the four-momentum of the detected hadron. The PM approximation of the process is given displayed in Fig.~\ref{fig:SIDIS_PM}.
\\
\\
SIDIS data are available for different kind of targets and species of hadrons identified in the final state (p, D, He and $\pi^{\pm}$, $h^{\pm}$, respectively). Since frag- mentation functions depend strongly on the flavour of the originating par- ton, and on the charge of the hadron, it is possible to construct alternative combinations of parton distributions which are independent from the ones contributing to the inclusive asymmetry. In this way, in principle it possible to do a full flavour decomposition for the polarized parton distributions.
\\
For collisions with longitudinally polarised leptons and protons, the polarised the structure function can be isolated by taking difference between the cross-sections with opposite target helicities [\href{https://arxiv.org/abs/2109.00847v2}{arXiv:2109.00847}] 
%%
\begin{equation}
  \frac{d \Delta \sigma^h}{dx \, dy \, dz} \equiv  \frac{1}{2} \left(\frac{d^3\sigma_{h}^{\uparrow \Uparrow}}{dx\,dy\,dz} - \frac{d^3\sigma_{h}^{\uparrow \Downarrow}}{dx\,dy\,dz}\right) = \frac{4\pi \alpha_{em}^2}{Q^2} (2-y) g_{1}^{h} (x,\,z,\,Q^2)\,.
\end{equation}
%%
As for DIS, it can be proved (see \textit{e.g.} Collins) that a factorisation theorem holds also for SIDIS processes. The structure function $g_1^h (x,z,Q^2)$ may be written as
%%
\begin{equation}
  \begin{split}
    g_1^{h} (x,z,Q) & = \frac{1}{2} \sum_{q,\bar{q}} e_{q}^2 \left[ \Delta q(x,Q) \otimes \Delta \mathcal{C}_{qq}^{1} \otimes D_{q}^{h}(z,Q) + \right.\\
    & \left.\Delta q(x,Q) \otimes \Delta \mathcal{C}_{gq}^{1} \otimes D^{h}_{g}(z,Q) + \Delta g(x,Q) \otimes \Delta \mathcal{C}_{qg}^{1} \otimes D^{h}_{q}(z,Q) \right] \,,
    \end{split}
    \label{eq:g1h}
\end{equation}
%%
where we have introduced the double convolution defined as 
%%
\begin{equation}
  f \otimes \Delta \mathcal{C} \otimes h (x,z) = \int \frac{d\xi}{\xi} \int \frac{d \zeta} {\zeta} f \left( \frac{x}{\xi}\right)  \, \Delta \mathcal{C} \left(\xi, \zeta\right) \, h \left( \frac{z}{\zeta} \right) \,.
\end{equation}
%%
In a certain sense, the FF "selects" the single flavour in the initial state. This has important phenomenological consequence in the parton determination, as we shall discuss in Chap.~\ref*{ch:3}.\par
Polarised coefficient functions have been computed at NLO in Ref.~[\href{https://arxiv.org/abs/hep-ph/9711387v1}{arXiv:hep-ph/9711387}] and up to (approximate) NNLO so far in Ref.~[\href{https://arxiv.org/abs/2109.00847v2}{arXiv:2109.00847}].\par
Phenomenologically, the quantity that is measured in experiments is a normalised asymmetry, similar to Eq.~\eqref{eq:A1}, and takes the form \href{https://arxiv.org/abs/hep-ph/0007068}{arXiv:hep-ph/0007068}
%%
\begin{equation}
  A_1^{h}(x,Q^2) \simeq  \frac{\int_{Z} dz \, g_1^h (x,z,Q^2)}{\int_{Z} dz \, F_1^h (x,z,Q^2)} \,,
\end{equation}
%%
where $Z$ denotes the kinematical region covered by final state hadrons.\par
Finally, we must remark that,as for PDFs, FFs contain a $Q^2$-dependence which arises from the factorisation of the UV divergences. The perturbative dependence on the scale $Q^2$ is given by the same DGLAP equations, Eqs.~\eqref{eq:DGLAP_coupled}, as for the parton densities
%%
\begin{equation}
  \frac{\partial}{\partial \mu^2} D_{q_i}^{h} (z,\mu^2) = \frac{\alpha_s(\mu^2)}{2\pi} \sum_{j} \int_{z}^{1} \frac{du}{u} P_{ji}\left( u, \alpha_s(\mu^2) \right) \, D_{q_j}^h \left( \frac{z}{u}, \mu^2 \right) \,.
\end{equation}
%% 
The LO splitting function for FFs have been computed in Ref.~[Phys. Lett. B76 (1978) 85, Phys. Lett. B79 (1978) 97, Nucl. Phys. B136 (1978) 445], and they agree with the
LO space-like DGLAP splitting functions. The NLO splitting function have been computed in [Nucl. Phys. B175 (1980) 27, Phys. Lett. B97 (1980) 437], and NNLO in [Phys. Lett. B638 (2006) 61, Phys. Lett. B659 (2008) 290, Nucl. Phys. B854 (2012) 133].\par
In conclusion, the theoretical framework for SIDIS is not so different from that of DIS. The only difference being the double convolution as in Eq.~\eqref{eq:g1h}, introducing an additional complexity for the numerical implementation, which will be addressed in next chapter.
\chapter{Conclusions}
\label{ch:5}

Polarised parton distribution functions play a key role in the context of perturbative QCD, and enter the calculations involved in any phenomenological study of hard scattering processes with polarised hadrons in the initial state. In this Thesis, I presented the first determination of spin-dependent parton distribution functions of the proton at next-to-next-to-leading order accuracy in pQCD.%

Experimental data of polarised processes are both less abundant and less precise than those available in the unpolarised counterpart, and polarised parton distributions are then known with much less precision. However, as the quality of the experimental information improves, it becomes more and more necessary to develop a solid framework that is able to face future data and that allows for more accurate determinations. The Electron-Ion Collider, which is expected to operate in the 2030s, will increase the precision of measurements and enlarge the kinematic coverage of data, rather limited at the present as shown in \secref{sec:4.2}. On the one hand, a wider kinematic range would reduce PDF uncertainties in the extrapolation regions. On the other hand, higher precisions will require more accurate predictions, possible by increasing the perturbative order in the calculations of observables. Hence, even though providing the first determination of polarised PDFs at NNLO accuracy has been the major goal of the present Thesis, it also wishes to provide a benchmark for the suitability of the presented framework for future data. Moreover, the results presented in this Thesis may suggest the way in which future measurements should be performed at the EIC in order to increase the quality of PDF determinations.%

The analysis has been carried out within the MAP methodology, inspired by the framework developed by the NNPDF Collaboration. The statistical tools adopted in this Thesis include Monte Carlo methods for error propagation, neural network for PDF parametrisation, and stochastic gradient descent for parameter optimisation. The determination is based on available data from inclusive and semi-inclusive Deep-Inelastic Scattering. The main results presented in this Thesis are here summarised:
%
\begin{itemize}
  \item I determined the \texttt{MAPpol1.0} set of helicity-dependent PDFs at NNLO accuracy in perturbative QCD, based on inclusive and semi-inclusive DIS data. The impact of higher order corrections on valence distributions is negligible, while sea quark distributions at NNLO differ in some respects. With regard to the gluon, it also acquires small changes when extracted at NNLO, still remaining highly unconstrained by the included data. Even though the global $\chi^2$ of the NNLO fit indicates a very good agreement between predictions and data, it has been observed a deterioration of the fit quality w.r.t. the NLO fit. I showed that this degradation depends on the data sets included in the analysis, and can be due to the limited precision of polarised measurements. Nevertheless, this situation might change once EIC will provide new polarised data sets with increased kinematic coverage and accuracy.
  \item The determination has been carried out by means of two theoretical constraints - sum rules and positivity. I verified that these assumptions do not introduce theoretical biases in the analysis, and that the results remain stable upon variation of these conditions. Indeed, I showed that the functional space of PDFs allowed by data is compatible, within uncertainties, with that of the theoretical constraints.
  \item I compared the \texttt{MAPpol1.0} set at next-to-leading order accuracy with other available sets \cite{Nocera:2014gqa, Ethier:2017zbq, deFlorian:2009vb} at the same perturbative order. In general, the PDF set presented in this Thesis is compatible, within uncertainties, with the other determinations, yet some differences emerge in some respects. As extensively discussed, these are mainly due to the different experimental information and methodologies that each collaboration adopts to carry out the analysis.
\end{itemize}
%

\section{Outlook and future directions}
%
The \texttt{MAPpol1.0} polarised parton set is based on inclusive and semi-inclusive DIS measurements. Still, additional information on the spin structure of the proton can be achieved by including a range of other processes to the analysis, as frequently suggested throughout this Thesis.%

The inclusion of polarised Drell-Yan measurements may be beneficial to both precision and accuracy of quark and antiquark distributions. This would require the implementation of the theoretical framework needed to compute polarised Drell-Yan predictions in the global analysis, similar to what has been done in \chapref{ch:2} for DIS and SIDIS. Currently, NLO corrections to DY cross-sections have been computed in Refs.~\cite{Gehrmann:1997ez,Bonvini:2010tp}, while at NNLO they have been recently computed in Ref.~\cite{Boughezal:2021wjw}. Alternatively, polarised Drell-Yan can be taken into account by means of the Bayesian reweighting and unweighting method \cite{Ball:2010gb, Ball:2011gg}, as done in \texttt{NNPDFpol1.1}.%

In addition, jet production from proton-proton collisions would constrain the gluon distribution at leading order thanks to gluon-initiated processes. Jet production data can be implemented only through Bayesian reweighting (as in \texttt{NNPDFpol1.1}), since a perturbative calculation for such a process is not yet possible.%

In conclusion, all these further constraints can be implemented in the analysis presented in this Thesis to achieve a more comprehensive scenario that includes most of the experimental information currently available. In the long term, EIC data are expected to reduce the discrepancy between polarised and unpolarised data, and thus allowing for a more reliable determination of spin-dependent parton distribution functions.
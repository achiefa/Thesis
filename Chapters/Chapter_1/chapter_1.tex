\chapter{Introduction}
\label{ch:1}

The quest to unravel the structure of hadrons, the family of composite particles bound by the strong force, has been a captivating journey that spans several decades in the history of particle physics. Quantum Chromodynamics (QCD) provides a framework that describes the structure of hadron in terms of colour-charged fundamental constituents -- quarks and gluons, commonly called \textit{partons}. These particles are \textit{confined} into hadrons and, at low energy, they are not observed isolated. The reason being that the energy of the mutual interactions grows with the separation between the colour charges. In order to win the strong bounds and probe the hadron structure, experiments must be performed at high energy scales. In this limit, QCD  behaves as \textit{asymptotically free}, and it becomes possible to apply perturbative technologies in QCD calculations. Perturbative QCD (pQCD) allows then to compute predictions that are compared to experiments with extraordinary accuracy.%

Being composite particles, hadrons have a composite spin that arises from the spins of their constituents. Although the value of the hadron spin can be accessed through experiments, a rigorous description of how partons and their orbital angular momenta combine is an intriguing challenge.%

Understanding the structure of hadrons and the origin of their spin is a crucial aspect in particle physics, as it provides insights in the fundamental interactions that govern the behaviour of matter at smallest scales. For this reason, research in this area continues through a combination of theoretical studies, experimental investigations, and advanced computational techniques. 


\section{A whirlwind review of the nucleon structure}

It may be reasonable to trace the beginning of hadrons' history back to 1911, when the Rutherford's famous scattering experiments revealed the presence of positive charge particles - protons - in the core of atoms. Two decades later, James Chadwick discovered the neutral twin of the proton, that is the neutron. The first relevant model that accounted for the force the held together these two particles within the nucleus was proposed by Hideky Yuakawa in 1934. According to this theory, the force between nucleons is mediated by a massive particle, similarly to what photon does in QED interactions. This particle was identified as the pion in 1947 from cosmic rays experiments. From the same experimental source and almost simultaneously, strange particles such as kaons and lambda particles were discovered. This first period marked just the start of what would be a long journey in new incoming discoveries. Over years, the number of identified hadrons rapidly increased, and it became necessary to provide a description of the observed plethora of strong interacting particles.%

The first attempt to bring some order in the chaos of the \quotes{particle zoo} was due to Murry Gell-Mann in 1961, who introduced the so-called \textit{Eightfold Way}. This model classified hadrons within a geometrical pattern, in keeping with their features (charge and strangeness). However, it did not explain why particle should be classified in that manner. In order to cope this questioning puzzle, Murray Gell-Mann and George Zweig independently proposed the static (non-relativistic) \textit{quark model} in the early 1960s. This model introduced the idea of point-like constituents as building blocks of matter, in that hadrons are visualised as made up of spin one-half particles (\textit{i.e.} quarks) with fractional charge. The quark model postulated that these fundamental constituents come with in three types or \textit{flavours}, namely \textit{up}, \textit{down}, and \textit{strange}. This theoretical framework, supported with the rules of spin composition, was not solely able to classify the known hadrons, as the Eightful Way already did, but it was also able to predict potentially new unobserved particles.%

The last ingredient, which was kind of necessary to account for the Pauli exclusion principle and to provide a comprehensive description, was the contribution of O. W. Greemberg in 1964, who postulated that each quark flavour come in three different colours (\quotes{red}, \quotes{green}, and \quotes{blue}). In accordance to the exclusion principle, all naturally occurring particles are colourless, that is in a combination of quarks whose colours simplify each other. Incidentally, this feature was able to explain why only some combinations of quarks and antiquarks were observed, but not all of them.%

\subsection*{Deeply-Inelastic experiments}

The quark model successfully explained the existence of various hadron states and their properties. However, it suffered from one fundamental aspect, which gave origin to a widespread scepticism. Indeed, quarks had never been observed, nor experiments were able to produce them. Scientists referred to this phenomenon as \textit{quark confinement}, though the subtle mechanism behind that was not clear.

The compelling about the presence of elementary constituents inside hadrons fostered experiments to explore the proton's internal structure with improved resolutions. Similarly to the Rutherford's apparatus, Deep Inelastic Scattering (DIS) were used to fire high energy beams of particles (usually leptons) into target protons. Experiments of this type were first performed using high-energy electrons at the Stanford Linear Accelerator Center (SLAC) in late 1960s. In this occasion, structure functions were observed scale invariant, that is they exhibited a dependence on dimensionless kinematic quantities rather than on the absolute energy scale of the process. Such a discovery, knwon as Bjorken scaling (1968), implied that the proton structure is independent of the absolute resolution scale, and hence reveals a point-like substructure. Feynman proposed to call these elementary particles \textit{partons}, in order to distinguish the static picture of the quark model from the dynamical behaviour observed in DIS experiments.%

For a long time DIS continued to shed light on the hadron structure, strengthening the idea of point-like constituents inside hadrons. The parton model, developed by Richard Feynman and others, described quarks and gluon as \textit{partons} that carry fractional momentum inside hadrons. However, calculations in perturbative theories predicted that Bjorken scaling could not hold exactly and that it could be affected by small corrections in keeping with the scale of the process. Indeed, mild violation of the Bjorken scaling were observed experimentally. They could only be described through an interacting theory whose coupling constant approaches zero as the energy scale increases - a theory that behaves as \textit{asymptotically free}. The theoretical and experimental efforts converged into the theory of QCD, proposed by David Gross, David Politzer and Frank Wilczek in 1973. The quantum field theory of quarks and gluons is the modern framework to describe strong interactions and that naturally introduce \textit{colour} as fundamental quantity of the non-abelian gauge symmetry group $SU(3)$, which accounts for the asymptotic free nature.


\subsection*{Spin structure of hadrons}

Deep inelastic lepton-hadron scattering (DIS) has set the stage for our present understanding of the sub-structure of hadrons. Since then, immense experimental and theoretical efforts have been made to shed light on the nucleon structure and to provide a coherent description in terms of elementary particles.%

A complementary and equally important insight into the structure of the nucleon was provided by polarised DIS, which allows for the investigation of the spin content in hadrons. Initially, it was believed that quarks carried most of the hadron's spin, that is
%%
\begin{equation}
  S(Q^2) = \sum_{f} \bra*{P,S} \hat{J}_f^z(Q^2) \ket*{P,S} = \sum_f S^z_f (Q^2) = \frac{1}{2} \,,
\end{equation}
%
where the sum runs over quark and antiquark flavours ($u,d,s,\bar{u}, \bar{s}, \bar{s}$). The expression refers to the spin of a hadron moving along the $z$-axis and spin aligned to the direction of motion. The result provided by the European Muon Collaboration in the 1988~\cite{EuropeanMuon:1989yki} delineated a slightly different picture. Indeed, the measured fraction of spin carried by quarks along the quantisation axis, $\left< S_z \right>_{\T{quarks}}$, which accounted for
%%
\begin{equation}
  \left< S_z \right>_{\T{quarks}} (Q^2 = 10.7 \, \T{GeV}^2) = 0.060 \pm 0.047 \, (\T{stat}) \pm 0.069 \, (\T{sys}) \,,
\end{equation}
%%
clearly suggested that quarks carry a small portion of nucleon's spin, which implied a \quotes{spin crisis in the parton model}~\cite{Leader_Anselmino}. It was believed that large polarised gluon contributions could resolve the crisis. However, subsequent measurements, supplied with QCD analysis, provided a polarised gluon density which is too small to account for the total spin~\cite{Leader:2005ci}. Indeed, a more reliable description must take into account the gluon contribution as well as the orbital angular momenta that arise therein. Hence, the spin decomposition should read\footnote{\footnotesize It is worth mentioning that the expression for the spin decomposition depends on the reference frame in which the decomposition is carried out. In particular, the one written above is known as Jeffe-Manohar spin decomposition and holds in the Infinite Momentum Frame $P^z \rightarrow \infty$ and in the Light-cone gauge $A^+=0$.}
%
\begin{equation}N
  \frac{1}{2} \Delta \Sigma(Q^2) + \Delta G(Q^2) + \ell_{q}(Q^2) + \ell(Q^2) = \frac{1}{2} \,
  \label{eq:AM_decomposition}
\end{equation}
%
where $\Delta \Sigma$ and $\Delta G$ represent the spin contribution of quarks and gluons, respectively. These quantities may be determined by performing global fit to data within the framework of QCD\footnote{\footnotesize It must be observed that recent developments in lattice QCD are addressing to problem of computing first moment of the distributions without performing any global fit to data. However, such technology, as yet, only exists in embryonic form.}. In doing so, one is able to extract the so-called helicity dependent parton distribution functions (PDFs), which enclose the information of the hadron structure in terms of polarised partons. The knowledge of such quantities allows one to compute the intrinsic spin carried by quarks and gluon, that is $\Delta \Sigma$ and $\Delta G$. However, the orbital contributions that appear in Eq.~\eqref{eq:AM_decomposition} cannot be constrained by data so far, but may be inferred by difference from Eq.~\eqref{eq:AM_decomposition}.

\subsection*{The Electron-Ion Collider era}

Since the advent of the so-called \quotes{spin crisis} in the 80s, several efforts have been made to enhance the available experimental information in for polarised processes. This occurred in different facilities such as CERN, SLAC, DESY and Jefferson Lab. Despite that, the available data on helicity dependent processes remains scarce. If compared to the unpolarised counterpart, not only the kinematic covered by data is smaller, but also the information is less accurate. As a consequence, the quality of global fits for helicity dependent PDFs is limited by this restricted experimental availability. 

The Electron-Ion Collider (EIC) at Brookhaven National Laboratory~\cite{AbdulKhalek:2021gbh}, which is expected to operate in the 2030's, 
will provide new data with and extended kinematic coverage and an enhanced quality of the measurements. With polarised beams of both leptons and hadrons, the EIC is expected to give new insights into the hadron structure. The potential impact of EIC data on polarised PDFs have been argued in several studies (see \textit{e.g.} Refs.~\cite{Borsa:2020lsz, Aschenauer:2012ve, Ball:2013tyh, Aschenauer:2015ata}), assessing the improvements on both accuracy and precision sides. These studies have been carried out by means of pseudo-data sets generated with different values of the center-of-mass energy, spanning a range that goes from $\sqrt{s} = 44.4$ to $\sqrt{s} = 141.4 \,\T{GeV}$. Although the results of these analyses may slightly differ from each other, the overall pictures that emerges from the inclusion of EIC data may be summarised as follows:
%
\begin{itemize}
  \item EIC data would reduce PDF uncertainties, in particular in the low-$x$ region where data are currently scarce. The impact on quark distributions is moderate, but drastic reductions of the uncertainties for the gluon are expected.
  \item The improvements in parton densities would also have an impact on the first moment of the distributions, hence on the spin decomposition Eq.~\eqref{eq:AM_decomposition}.
  \item Incidentally, EIC data may improve the precision of PDFs in the extrapolation regions, as shown in Ref.~\cite{Ball:2013tyh}.
\end{itemize}
%
Nevertheless, these premature results will only be verified once the EIC will start out. For the moment, we can only tweak our fitting technologies in order to be ready for future upcoming data.

\section{Outline of the thesis}

The experimental progresses must be followed by theoretical and methodological improvements. At the present time, the available polarised PDF sets have been determined only at next-to-leading order in pQCD~\cite{deFlorian:2008mr, Ethier:2017zbq, Nocera:2014gqa}. In order to match the precision of EIC data, predictions must be more accurate, and it becomes then mandatory to include higher order corrections in global analyses. Recent theoretical developments have been made in pQCD, providing the expressions of perturbative terms up to nerxt-to-next-to-leading order, for polarised inclusive and semi-inclusive (SIDIS) DIS, as well as Drell-Yan processes.%

This Thesis aims to present the first determination of polarised PDFs at next-to-next-to-leading order accuracy in pQCD, using data from polarised DIS and SIDIS. Incidentally, the work serves as a test to ensure the reliability of the fitting methodology for future EIC data.%

This Thesis will not consider higher-twist corrections, whose contribution is mitigated by proper kinematic cuts to the data set. Moreover, heavy flavours will be not taken into account, by acting on the parametrisation scale. Finally, the present determination only deals with longitudinal polarised PDFs and the effects of transverse distributions are beyond the scopes of this work.%

The Thesis is organised as follows:\\[10pt]
\begingroup
\textbf{Chapter \ref{ch:2}. Polarised Inclusive and Semi-Inclusive Deep-Inelastic Scattering}. I review the theoretical formalism necessary to describe both DIS and SIDIS experiments. The discussion will be restricted to processes mediated by a single virtual photon. Indeed, the available experimental information involves energies that do not exceed a few hundred of GeV, and does not include processes with neutrino beams. Thus, I will not consider contributions coming from neutral0current (SI)DIS mediated by a $Z$ boson, as well as charged-current (SI)DIS mediated by $W^{\pm}$ bosons. I derive the expression for the polarised cross-section in terms of polarised structure functions and I briefly review the phenomenological relations between structure functions and the measured observables. The effect of QCD corrections will be discussed, as well the evolution of parton densities with varying energy scales.
\\[5pt]
\textbf{Chapter \ref{ch:3}. Phenomenology of PDF determination.} I depict the general picture of PDF determination, focusing on the main obstacles that one encounters when dealing with global fits. Then, I will discuss how these problems are addressed in this Thesis, presenting the methodological and numerical tools used to carry out the PDF determination. As I will explain, the fitting methodology, hosted by the MAP collaboration, is inspired by the framework developed by the NNPDF Collaboration.
\\[5pt]
\textbf{Chapter \ref{ch:4}. Polarised PDFs from DIS and SIDIS.} In this chapter, the set of polarised parton distributions determined at NNLO is presented. I review the data sets included in the QCD analysis, and I provide details on the fitting strategy. Then, the impact of NNLO corrections is assessed, and the corresponding set is compared to other PDF sets available. Finally, the stability of the results upon the variation of some fitting settings is discussed.
\\[5pt]
\textbf{Chapter \ref{ch:5}. Conclusions.} I will draw the conclusion of the present work, summarising the results obtained in this Thesis. I also discuss future possible developments that may enhance the present determination.
\endgroup
\chapter{Introduction}
\label{ch:1}

The quest to unravel the structure of hadrons, the family of composite particles bound by the strong force, has been a captivating journey that spans several decades in the history of particle physics. At the present, the rich dynamic of strong interactions is described by Quantum Chromodynamics (QCD). It provides a framework for describing the structure of hadron in terms of colour-charged fundamental constituents -- quarks and gluons, grouped together under the name of partons. These particles have never been observed in isolation, and thus they are \textit{confined} into hadrons. The reason being that the energy of the mutual interactions grows with the separation between the colour charges. In order to win the strong bounds and probe the hadron structure, experiments must be performed at high energy scales. In this limit, QCD  behaves as \textit{asymptotically free}, and it becomes possible to apply perturbative technologies in QCD calculations. Perturbative QCD (pQCD) allows then to compute predictions that are compared to experiments with extraordinary accuracy.%

Being composite particles made up of quarks and gluons, hadrons have a composite spin that arises from the spins of their constituents. Although the value of the hadron-spin can be accessed through experiments, a rigorous description of how partons and orbital angular momentum combine to contribute to the overall hadron-spin is an intriguing challenge, far from being successfully achieved.%

Understanding the structure of hadrons and the origin of their spin is a crucial aspect of particle physics, as it provides insights into the fundamental forces and interactions that govern the behaviour of matter at the smallest scales. For this reason, research in this area continues through a combination of theoretical studies, experimental investigations, and advanced computational techniques. 


\section{A whirlwind review of the nucleon structure}

It is possible to trace the discovery of hadrons back to 1911, when the Rutherford's famous scattering experiments revealed the presence of positive charge particles in the core of atoms, for which was given the name proton by Rutherford himself. In 1932, James Chadwick discovered the neutral twin of the proton, that is the neutron. It was necessary to explain the observed mass of helium, which contains only two protons that do not account for the total weight. This period opens the path to future experimental discoveries and theoretical turning points.%

In 1934, Yukawa proposed the first relevant theory of the strong force. Based on this model, nucleons were held together inside the nucleon by means of a quantised field, whose quanta came to be known as the meson. This particle had never been observed until 1973, when two separate groups (Anderson \& Neddermeyer and Street \& Stevenson) identified new particles from cosmic rays, whose properties seemed to match those of Yukawa's particles of strong force. However, it was only in 1946 that experiments were able to discriminate two different components inside cosmic rays, namely the pion and the muon. In that regard, only the latter was the Yukawa meson. One year later, strange particles were observed in cosmic rays and were eventually identified as hadrons. The discoveries of these first hadrons, namely the protons, the neutrons, pions, and kaons marked just the start of a richer background (\red{I don't like this word. Change it!}). Indeed, the organised and ordered picture that enclosed these first hadrons, was perturbed as the number of new hadrons drastically increases by 1960 and beyond. The plethora of strong interacting particles was divided into two big families, namely mesons and baryons. But other than this label, a formal description of the \quotes{particle zoo} that had been observed to arise was not clear.%

The first attempt to make some order in the chaos of particle physics was due to Murry Gell-Mann in 1961, who introduced the so-called \textit{Eightfold way}. This model classified hadrons within a geometrical pattern, in keeping with their features (charge and strangeness). However, this model only provided a way to arrange particles, but did not explain why particle should be classified in that manner. In order to cope this questioning puzzle, Murray Gell-Mann and George Zweig independently proposed the static (non-relativistic) \textit{quark model} in the early 1960s. This model introduced the idea of point-like constituents as building blocks of matter, in that hadrons are visualised as made up of spin one-half particles with fractional charge, called quarks. The quark model postulated that these fundamental constituents come with in three types or \textit{flavours}, namely up, down, and strange. With that background4, supported by the rules of spin composition, it was not solely possible to classify the known hadrons, as the Eightful way did, but also to predict potentially new unobserved particles.%

The last ingredient, which was kind of necessary to account for the Pauli exclusion principle and provide a comprehensive description, was the contribution of O. W. Greemberg in 1964, who postulated that each quark flavour come in three different colours (\quotes{red}, \quotes{green}, and \quotes{blue}). In accordance to the exclusion principle, all naturally occurring particles are colourless, that is in a combination of quarks whose colours simplify each other. Incidentally, this feature was able to explain why only some combinations of quarks and antiquarks were observed, but not all of them.%

\subsection*{Deeply-Inelastic experiments}

The quark model successfully explained the existence of various hadron states and their properties. However, it suffered from one fundamental aspect, which gave origin to a widespread scepticism about the quark model. Indeed, quarks had never been observed, nor experiments were able to produce them. Scientists referred to this phenomenon as \textit{quark confinement}, though the subtle mechanism behind that was not yet clear.

The question about the presence or not of elementary constituents inside hadrons fostered experiments to explore the proton's internal structure. Similarly to what Rutherford did in his famous ... , Deep Inelastic Scattering (DIS) were used to fire high energy beams of particles (usually leptons) into target protons. Experiments of this type were first performed using high-energy electrons at the Stanford Linear Accelerator Center (SLAC) in late 1960s, when structure functions for inelastic processes were observed to be (approximatively) scale invariant, that is they do not depend on the absolute energy of the experiment but on dimensionless kinematic quantities. Bjorken scaling (1968) implied that the proton structure is independent of the absolute resolution scale, and hence reveals a point-like substructure. Feynman proposed to call these elementary particles \textit{partons}, in order to ... the static structure proposed by the quark model and the dynamical behaviour observed in DIS experiments.%

For a long time DIS continued to shed light on the hadron structure, strengthening the idea of point-like constituents inside hadrons. The parton model, developed by Richard Feynman and others, described quarks and gluon as \textit{partons} that carry fractional momentum inside hadrons and relied on the assumption of exact $SU(3)$ flavour symmetry. However, calculations in perturbative theories predicted that Bjorken scaling could not hold exactly and that it could be affected by small corrections in keeping with the scale of the process. Indeed, mild violation of the Bjorken scaling were observed experimentally could only be described by an interacting theory whose coupling constant approaches zero as the energy scale increases - a theory that behaves as \textit{asymptotically free}. The theoretical and experimental efforts converged into the theory of Quantum Chromodynamics (QCD), proposed by David Gross, David Politzer and Frank Wilczek in 1973. The quantum field theory of quarks and gluons is the modern framework to describe strong interactions and that naturally introduce \textit{colour} as fundamental quantity of the non-abelian gauge symmetry group $SU(3)$, which accounts for the asymptotic free nature. (also DGLAP)
\subsection*{Spin structure of hadrons}

Deep inelastic lepton-hadron scattering (DIS) has set the stage for our present understanding of the sub-structure of hadrons. Since then, immense experimental and theoretical efforts have been spent to shed light on the nucleon structure and to provide a coherent description in terms of elementary particles.%

A complementary and equally important insight into the structure of the nucleon was then provided by polarised DIS, which allows for the investigation of the spin structure of hadrons. In the context of the parton model, the spin of a hadron moving along the $z$-axis with helicity $h=1/2$ should emerge as the sum of the spin of the internal constituents:
%%
\begin{equation}
  S(Q^2) = \sum_{f} \bra*{P,S} \hat{J}_f^z(Q^2) \ket*{P,S} = \sum_f S^z_f (Q^2) = \frac{1}{2}\,.
\end{equation}
%
Where the sum runs over quark and antiquark flavours ($u,d,s,\bar{u}, \bar{s}, \bar{s}$).Hence, it was expected that the the bound state received its main contribute from quarks. In that respect, the historical result provided by the European Muon Collaboration in the 1988~\cite{EuropeanMuon:1989yki} delineated a slightly different pictures. Indeed, the experiment measured the fraction of proton spin carried by quarks along the quantisation axis, $\left< S_z \right>_{\T{quarks}}$, whose values was~\cite{EuropeanMuon:1989yki}
%%
\begin{equation}
  \left< S_z \right>_{\T{quarks}} (Q^2 = 10.7 \, \T{GeV}^2) = 0.060 \pm 0.047 \, (\T{stat}) \pm 0.069 \, (\T{sys}) \,.
\end{equation}
%%
This suggested that quarks carry a small portion of nucleon's spin, which implied a \quotes{spin crisis in the parton model}~\cite{Leader_Anselmino}. It was believed that large polarised gluon contribution could resolve the crisis. However, subsequent measurement provided a polarised gluon density which is too small to account for the total spin. 

Since the advent of the so-called "spin crisis" in the 80s \cite{EuropeanMuon:1987isl}, several efforts have been made  to measure of polarised inclusive deeply inelastic processes in lepton-nucleon collisions at labs like CERN, SLAC, DESY and Jefferson Lab. 

In fact, considerable effort has been spent after the results of the European Muon Collaboration (EMC) in the 1980 \cite{EuropeanMuon:1989yki}. Although the interpretation of the results were not completely correct, the measurement clearly indicated that only a part of the proton's spin is carried by the valence quark, leading to the so-called "spin crisis" in constituent quark model. The discrepancy between theory and data was then resolved by the inclusion of other sources in the decomposition of the proton spin, such as the sea quarks and the gluon spin contributions, the latter heuristically predicted in the PM, then naturally introduced by QCD.\par

\subsection*{The Electron-Ion Collider era}


\section{Outline of the thesis}

Global fit determinations of parton distributions relies on two main ingredients that equally contribute to the analysis -- theory and data. On the one hand, developments on pQCD allows computing more accurate predictions, for instance introducing higher order corrections. On the other hand, theoretical predictions must be compared to data. The quality of the experimental information propagates through the analysis, and reflects on the quality of the fit.%

Despite the experimental effort for enhancing the experimental information in terms of both accuracy and abundance, the availability of experimental information on helicity dependent processes remains scarce. Not only the kinematic coverage is limited, but also the information is less accurate than the unpolarised case. As a consequence, the reliability of polarised PDFs is highly limited. The Electron-Ion Collider at Brookhaven National Laboratory \cite{AbdulKhalek:2021gbh} will provide new data with an extended kinematic coverage and a reduction of the uncertainties, even for the polarised case. The potential impact on helicity dependent PDFs have been argued in recent studies (see \textit{e.g.} Ref.~\cite{Borsa:2020lsz}), assessing the impact on both accuracy and precision.%

EIC data will require more accurate predictions in order to match their precisions. The inclusion of higher-order corrections inside global fit would then become mandatory to achieve this accuracy. This Thesis presents the first determination of polarised PDFs at next-to-next-to-leading order accuracy in pQCD. The final result will also assess the stability of the methodological framework developed during this project, in order to be prepared for the incoming experimental improvements.   
- What is the aim of this Thesis
- How do we obtain the results
- What this Thesis does not take into account
- Structure of the Thesis
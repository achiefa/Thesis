\chapter{Introduction}
\label{ch:1}

Quantum Chromodynamics (QCD) is the field theory that describes the interaction of colour-charged fundamental particles -- quarks and gluons, commonly called \textit{partons}. These particles make up all the hadrons, and they cannot be observed in isolation. The reason is that the energy of the QCD interaction grows with the separation between the colour charges. In order to break the strong bounds and probe the hadron structure, experiments must be performed at high energy. In this limit, QCD  behaves as \textit{asymptotically free}, and it becomes possible to compute cross-sections as a perturbative series in the strong coupling. Predictions are then compared to measurements with increasing accuracy.%

The quest to unravel the structure of hadrons, the family of composite particles bound by the strong force, has been a captivating journey that now spans several decades in the history of particle physics. Being composite particles, hadrons have a composite spin that arises from the spin of their constituents. Although the hadron spin can be accessed through experiments, a rigorous description of how partons and their orbital angular momenta combine is an intriguing challenge.%

Understanding the structure of hadrons and the origin of their spin is a crucial aspect of particle physics, as it provides insights into the fundamental interactions that govern the behaviour of matter at small scales. For this reason, research in this area continues through a combination of theoretical studies, experimental investigations, and advanced computational techniques. 


\section{A brief review of nucleon structure}

Investigations into hadron structure can be traced back to 1911, when Rutherford's famous scattering experiment revealed the presence of positive charged particles -- protons -- in the core of atoms. Two decades later, James Chadwick discovered the neutral twin of the proton, eventually named neutron. The first relevant model that accounted for the force that held together these two particles within the nucleus was proposed by Hideky Yukawa in 1934. According to Yukawa theory, the force between nucleons is mediated by a massive particle -- similarly to what occurs in QED -- which was identified as the pion in 1947 from cosmic ray experiments. From the same experimental source and almost simultaneously, strange particles such as kaons and lambda particles were discovered. This first period marked just the start of what would have been a long journey towards new incoming discoveries. As the number of identified hadrons increased over years, it became more and more necessary to describe the \quotes{particle zoo} in terms of a reliable model.%

The first attempt to bring order in the plethora of observed hadrons was due to Murry Gell-Mann in 1961, who introduced the so-called \textit{Eightfold Way}. This model classified hadrons within a geometrical pattern in accordance with their intrinsic properties (charge and strangeness), although it did not explain why particle should be classified in that manner. In order to solve this puzzle, Murray Gell-Mann and George Zweig independently proposed the static (non-relativistic) \textit{quark model} in the early 1960s. This model introduced the idea of point-like constituents as building blocks of matter, in that hadrons are visualised as made up of spin one-half particles (\textit{i.e.} quarks) with fractional charge. The quark model postulated that these fundamental constituents came in three types or \textit{flavours}, namely \textit{up}, \textit{down}, and \textit{strange}. This theoretical framework, supported with the rules of spin composition, was not solely able to classify the known hadrons, as the Eightful Way already did, but it also predicted potentially new unobserved particles.%

The last ingredient, which was kind of necessary to account for the Pauli exclusion principle and to provide a comprehensive description, was the contribution of O. W. Greemberg in 1964, who postulated that each quark flavour could come in three different colours (\quotes{red}, \quotes{green}, and \quotes{blue}). Thus, in accordance with the exclusion principle, all naturally occurring particles are colourless, that is they are made of quarks whose overall colour is neutral. Incidentally, this feature was able to explain why some combinations of quarks and antiquarks were observed, and some were not.%

\subsection*{Deeply-Inelastic experiments}

The quark model successfully explained the existence of various hadron states and their properties. However, it suffered from one fundamental aspect, which gave origin to a widespread scepticism. Indeed, quarks had never been observed, nor experiments were able to produce them. Scientists referred to this phenomenon as \textit{quark confinement}, though the subtle mechanism behind that was not clear.

The compelling question about the presence of elementary constituents inside hadrons fostered experiments to explore the proton's internal structure with increasing resolution. Similarly to the Rutherford's experiment, Deep Inelastic Scattering (DIS) was used to scatter high energy beams of particles (usually leptons) off target protons. Experiments of this type were first performed using high-energy electrons at the Stanford Linear Accelerator Center (SLAC) in late 1960s. These experiments showed that structure functions were approximately scale invariant, that is they exhibited a dependence on dimensionless kinematic quantities rather than on the absolute energy scale of the process. Such a discovery, known as Bjorken scaling (1968), implied that the proton structure was independent of the absolute resolution scale, hence revealing a point-like substructure. Feynman proposed to call these elementary particles \textit{partons}, in order to distinguish the static picture of the quark model from the dynamical behaviour observed in DIS experiments.%

Ever since DIS continued to shed light on hadron structure, strengthening the idea of point-like constituents inside hadrons. The parton model, developed by Richard Feynman and others, described quarks and gluon as \textit{partons} that carry fractional momentum inside hadrons. However, calculations in perturbative theories predicted that Bjorken scaling could not hold exactly and that it could be affected by small corrections related to the scale of the process. Indeed, mild violations of the Bjorken scaling were observed experimentally. They could only be described through an interacting theory whose coupling constant approached zero as the energy scale increased -- a theory that behaved as \textit{asymptotically free}. The theoretical and experimental efforts converged into the theory of QCD, proposed by David Gross, David Politzer and Frank Wilczek in 1973. The quantum field theory of quarks and gluons is the modern framework to describe strong interactions. QCD naturally introduces \textit{colour} as fundamental quantity of the non-abelian gauge symmetry group $SU(3)$, which accounts for the asymptotic free nature of the interaction.


\subsection*{Spin structure of hadrons}

Deep inelastic lepton-hadron scattering has set the stage for our present understanding of the sub-structure of hadrons. Since then, significant experimental and theoretical efforts have been made to shed light on the nucleon structure and to provide a coherent description in terms of elementary particles.%

A complementary and equally important insight into the structure of the nucleon was provided by polarised DIS, which allows for the investigation of the spin content of hadrons. Initially, it was believed that quarks carried most of the hadron spin, that is
%%
\begin{equation}
  S(Q^2) = \sum_{f} \bra*{P,S} \hat{J}_f^z(Q^2) \ket*{P,S} = \sum_f S^z_f (Q^2) = \frac{1}{2} \,,
  \label{eq:S(Q2)}
\end{equation}
%
where $P$ is the hadron momentum, $S$ is its spin, and $\hat{J}_f^z$ is the spin operator for each flavour $f$ along the $z$-axis. The sum runs over quark and antiquark flavours ($u,d,s,\bar{u}, \bar{s}, \bar{s}$). The expression refers to the spin of a hadron moving along the $z$-axis and spin aligned with the direction of motion. The kinematic variable $Q^2$ is the energy at which the proton structure is probed\footnote{\footnotesize The dependence on $Q^2$ is introduced by the DGLAP evolution equations, which will be defined later in \chapref{ch:2}.}. Still, the result provided by the European Muon Collaboration in 1988~\cite{EuropeanMuon:1989yki} showed a different picture. Indeed, the measured fraction of spin carried by quarks along the quantisation axis, $\left< S_z \right>_{\T{quarks}}$, which amounted to
%%
\begin{equation}
  \left< S_z \right>_{\T{quarks}} (Q^2 = 10.7 \, \T{GeV}^2) = 0.060 \pm 0.047 \, (\T{stat}) \pm 0.069 \, (\T{sys}) \,,
  \label{eq:EIC_spin}
\end{equation}
%%
clearly suggested that quarks carry a small portion of the nucleon spin, which implied a \quotes{spin crisis in the parton model}~\cite{Leader_Anselmino}.%

A full description of the nucleon spin should take into account the gluon contribution as well as the orbital angular momenta that arise therein. Hence, the complete spin decomposition should read\footnote{\footnotesize It is worth mentioning that the expression for the spin decomposition depends on the reference frame in which the decomposition is carried out. In particular, Eq.~\eqref{eq:AM_decomposition} is known as Jaffe-Manohar spin decomposition and holds in the Infinite Momentum Frame $P^z \rightarrow \infty$ and in the Light-cone gauge $A^+=0$.}
%
\begin{equation}
  \frac{1}{2} \Delta \Sigma(Q^2) + \Delta G(Q^2) + \ell_{q}(Q^2) + \ell(Q^2) = \frac{1}{2} \,
  \label{eq:AM_decomposition}
\end{equation}
%
where $\Delta \Sigma$ and $\Delta G$ represent the spin contribution along $z$ of quarks and gluons, respectively. These quantities are determined by performing global fit to data within the framework of QCD\footnote{\footnotesize It must be observed that recent developments in lattice QCD address the problem of computing first moments of distributions without performing any global fit to data. However, this does not give the possibility to extract polarised PDFs.}. In doing so, one is able to extract the so-called helicity-dependent parton distribution functions (PDFs), which bear information about the spin structure of hadrons. The knowledge of such quantities allows one to compute the intrinsic spin carried by quarks and gluon, that is $\Delta \Sigma$ and $\Delta G$. So far, the orbital contributions that appear in Eq.~\eqref{eq:AM_decomposition} cannot be constrained by data, but may be inferred by difference from Eq.~\eqref{eq:AM_decomposition}.%

However, it must be observed that a precise determination of the gluon contribution is extremely difficult to carry out. For instance, in inclusive (DIS) and semi-inclusive (SIDIS) deep inelastic scattering the gluon contribution enters only at NLO, and it is then suppressed by the running coupling (see \chapref{ch:2}). In addition, the limited kinematic coverage of polarised measurements makes the extraction of the gluon contribution even less precise. The inclusion of jet, di-jet, and hadron productions (which constrain the gluon even at LO) in the \texttt{DSSV}~\cite{DeFlorian:2019xxt} and \texttt{NNPDFpol1.1}~\cite{Nocera:2014gqa} analyses point to a non-negligible gluon polarisation (see Tab.~\ref{tab:spin_contribution}). Still, these values are affected by huge uncertainties, which prevent us to make any ultimate conclusion on the role of the gluon in the proton spin.
%%
\begin{table}[t]
  \centering 
  \small
  \begin{tabular}{c@{\hspace{1cm}} c c}
  \toprule \midrule
  \addlinespace
    &   $\Delta G$  &      $\Delta \Sigma$       \tabularnewline
  \midrule
  \addlinespace
    \texttt{NNPDFpol1.1}~\cite{Nocera:2014gqa}      &  $0.25 \pm 0.85$                 &  $0.24 \pm 0.10$               \tabularnewline
  \midrule
    \texttt{DSSV14}~\cite{DeFlorian:2019xxt}           &  $0.39 \pm 0.46$      &  $0.39 \pm 0.11$   \tabularnewline
  \midrule \bottomrule
\end{tabular}
  \caption{
    \small
    Quark and gluon contributions to the proton spin at $Q^2 = 10 \, \T{GeV}^2$ for \texttt{NNPDFpol1.1} and \texttt{DSSV14}. These values have been obtained by integrating polarised PDFs in the region $x \in [10^3,1]$, where $x$ is the Bjorken variable (defined in \chapref{ch:2}).
  \label{tab:spin_contribution}}
\end{table}
%%

\subsection*{The Electron-Ion Collider era}

Since the advent of the so-called \quotes{spin crisis} in the 80s, several efforts have been made to augment the quantity of experimental information in polarised processes. This occurred through different facilities around the world, such as CERN, SLAC, DESY and Jefferson Lab. Despite that, the available data on helicity-dependent processes remains scarce. If compared to the unpolarised counterpart, not only the kinematic region covered by data is smaller, but also the information is less precise. As a consequence, the quality of global fits for helicity-dependent PDFs is limited by this restricted experimental availability. 

The Electron-Ion Collider (EIC) at Brookhaven National Laboratory~\cite{AbdulKhalek:2021gbh}, which is expected to operate in the 2030s, will provide new data with extended kinematic coverage and enhanced precision. With polarised beams of both leptons and hadrons, the EIC is expected to give new insights into the nucleon spin structure. The potential impact of EIC data on polarised PDFs have been argued in several studies (see \textit{e.g.} Refs.~\cite{Borsa:2020lsz, Aschenauer:2012ve, Ball:2013tyh, Aschenauer:2015ata}), assessing the improvements on both accuracy and precision sides. These studies have been carried out by means of pseudo-data sets generated with different values of the center-of-mass energy $\sqrt{s}$, spanning a range that goes from $\sqrt{s} = 44.4$ to $\sqrt{s} = 141.4 \,\T{GeV}$. Although the results of these analyses slightly differ from each other, the emerging overall picture may be summarised as follows:
%
\begin{itemize}
  \item EIC data would reduce PDF uncertainties, in particular in the low-$x$ region where data are currently scarce. The impact on quark distributions is moderate, but drastic reductions of the uncertainties for the gluon are expected.
  \item The improvements in parton densities would also have an impact on the first moment of the distributions, hence on the spin decomposition Eq.~\eqref{eq:AM_decomposition}.
  \item Incidentally, EIC data may improve the precision of PDFs in the extrapolation regions, as shown in Ref.~\cite{Ball:2013tyh}.
\end{itemize}
%
Nevertheless, these predictions will only be verified once the EIC will start up. The tweak-and-test process that tunes our fitting technologies up for future upcoming data is the underlying scope that drew this Thesis.

%
\section{Outline of the thesis}

The experimental progresses must be followed by theoretical and methodological improvements. At the present time, the available polarised PDF sets have been determined only at next-to-leading order in pQCD~\cite{Nocera:2014gqa, deFlorian:2008mr, Ethier:2017zbq}. In order to match the precision of EIC data, predictions must be more accurate, and it becomes then mandatory to include higher order corrections in global analyses. Recent theoretical developments have been made in pQCD, providing expressions for perturbative terms up to next-to-next-to-leading order, regarding polarised DIS and SIDIS, as well as Drell-Yan production.%

This Thesis aims to present the first determination of polarised PDFs at next-to-next-to-leading order accuracy in pQCD, using data from polarised DIS and SIDIS. The machinery used to carry out this analysis makes use of modern computational techniques such as Monte Carlo methods for error estimation and neural network for PDF parametrisation. Incidentally, the work serves as a test to ensure the reliability of the fitting methodology for future EIC data.%

This Thesis will not consider higher-twist corrections, whose contribution is mitigated by proper kinematic cuts to the data set. Finally, the present determination only deals with longitudinal polarised PDFs and the effects of transverse distributions are beyond the scopes of this work.%

The Thesis is organised as follows:\\[10pt]
\begingroup
\textbf{Chapter \ref{ch:2}. Polarised Inclusive and Semi-Inclusive Deep-Inelastic Scattering}. I review the theoretical formalism necessary to describe both DIS and SIDIS experiments. The discussion will be restricted to processes mediated by single virtual photons. Indeed, the available experimental information covers energies that do not exceed a few hundred of GeV, and does not include processes with neutrino beams. Thus, I will not consider contributions coming from neutral current (SI)DIS mediated by a $Z$ boson, as well as charged-current (SI)DIS mediated by $W^{\pm}$ bosons. I derive the expression for the polarised cross-section in terms of polarised structure functions and I briefly review the phenomenological relations between structure functions and the measured observables. Effects of QCD corrections will be discussed, as well as the evolution of parton densities with varying energy scales.
\\[5pt]
\textbf{Chapter \ref{ch:3}. Phenomenology of PDF determination.} I depict the general picture of PDF determination, focusing on the main obstacles that one encounters when dealing with global fits. Then, I will discuss how these problems are addressed in this Thesis, presenting the methodological and numerical tools used to carry out the PDF determination. As I will explain, the fitting methodology, hosted by the MAP collaboration, is inspired by the framework developed by the NNPDF Collaboration.
\\[5pt]
\textbf{Chapter \ref{ch:4}. Polarised PDFs from DIS and SIDIS.} In this chapter, I present the set of polarised parton distributions determined at NNLO accuracy in pQCD. I review the data sets included in the QCD analysis, and I provide details on the fitting strategy. Then, the impact of NNLO corrections is assessed, and the corresponding set is compared to other PDF sets available. Finally, the stability of the results upon the variation of some fitting settings is discussed.
\\[5pt]
\textbf{Chapter \ref{ch:5}. Conclusions.} I will draw the conclusion of the present work, summarising the results obtained in this Thesis.
\endgroup